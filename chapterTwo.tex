
%% This file represents a sample second chapter of the main body of the dissertation
%%
%%**********************************************************************
%% Legal Notice:
%% This code is offered as-is without any warranty either
%% expressed or implied; without even the implied warranty of
%% MERCHANTABILITY or FITNESS FOR A PARTICULAR PURPOSE!
%% User assumes all risk.
%% In no event shall any contributor to this code be liable for any damages
%% or losses, including, but not limited to, incidental, consequential, or
%% any other damages, resulting from the use or misuse of any information
%% contained here.
%%**********************************************************************
%%
%% $Id: chapterTwo.tex,v 1.4 2006/08/24 21:12:59 Owner Exp $
%%


% A first, optional argument in [ ] is the title as displayed in the table of contents
% The second argument is the title as displayed here.  Use \\ as appropriate in
%   this title to get desired line breaks
\chapter[Background and Methodology]{Background and Methodology}

\section{Intel Optane DC Persistent Memory}

Persistent memory, also known as Non-volatile Memory (NVM), is a new addition to the memory/storage hierarchy shown in Figure 2 that fills the performance/capacity gap between DRAM and storage by combining traits of both worlds. Like DRAM, persistent memory comes in the form of Dual In-line Memory Modules (DIMMs) that reside on the memory bus. Therefore, applications can access persistent memory like they do with traditional DRAM, eliminating the need to page blocks of data back and forth between memory and storage. However, unlike DRAM DIMMs, persistent memory DIMMs offer greater capacity and can retain data when the system is shutdown or loses power. Thus, persistent memory can dramatically increase system performance and enable a fundamental change in computing architecture.

Intel Optane DC Persistent Memory Module (PMM) is the first commercially available persistent memory technology. This technology comes in DIMM form factor and embeds capacities up to 512GiB. Intel Cascade Lake processors are the first CPUs to support Intel Optane PMM. Like traditional DRAM, the Optane DIMM sits on the memory bus and connects to the processor's integrated memory controller (iMC). Figure 1 shows a typical system configuration of a hybrid node with DRAM and PMM. A user can have up to one Intel Optane DIMM per channel and up to six on a single socket providing capacities up to 3TiB per socket. Thus, an 8-socket system could access up to 24TB of persistent memory.

To ensure persistence, Intel Optane PMM sits within Intel’s asynchronous DRAM refresh (ADR) domain. Intel’s ADR domain ensures that CPU stores that reach the ADR domain will survive a power failure. The iMC maintains read and write pending queues (RPQs and WPQs) for each Optane DIMM and the ADR domain includes WPQs. Once the data reaches the WPQs, the ADR domain ensures that the iMC will flush the updates to persistent memory media on power failure.

The iMC communicates with the Optane DIMM using the DDR-T protocol in cache line access granularity (64B) (Figure 2). The memory access to NVDIMM arrive first at an Apache Pass Controller which coordinates access to the Optane Media. Similar to SSDs, the Optane DIMM perforsms address translation for wear-leveling and bad block management. Thus, it keeps an address indirection table (AIT) for this translation. 

The actual access to storage media occurs after address translation. Intel Optane DIMM physical media access granularity is 256 bytes. Thus, the Controller translates smaller requests into largest 256-byte accesses, causing write amplification as small stores become read-modify-write operations. The controller has a small write-combining buffer to merge adjacent writes.

Intel Optane PMem can operate in two modes: memory and App Direct. Memory mode uses Optane PMem as a large capacity main memory without persistence. DRAM is not visible to the users, and instead it serves as a cache for Optane PMem that is transparently managed by the operating system. In App Direct mode, Optane PMem DIMMs appear as independent, non-volatile storage devices. This allows Optane PMem to be used as a byte-addressable persistent memory that is mapped into the system physical address space and directly accessible by applications [].

% \section{Persistent Memory Development Kit}
% The Persistent Memory Development Kit (PMDK) is a collection of open-source libraries and tools that simplify managing and accessing persistent memory devices. Tuned for both Linux and Windows operating systems, these libraries build on the dax feature described on the SNIA NVM programming specification. The diagram on figure 5 describes the collection of libraries provided by PMDK. Although PMDK’s core libraries provide C APIs, higher level libraries such as pmemkv provide support for other programming systems.

\section{Serverless Computing}

Serverless computing is a cloud computing execution mode that enables developers to deploy their code without provisioning or managing server infrastructure. The term “serverless” is misleading, as servers are still being used by cloud providers to run the code for developers. However, instead of requesting and managing resources, developers simply provide their code, and the cloud providers handle the servers on behalf of their customers. Cloud providers are responsible for provisioning resources, scaling, fault tolerance, monitoring, security patches, and so on. Finally, developers simply pay by the execution time and resources used on their code invocations.

Function-as-a-service (FaaS) is the core compute engine for serverless computing. It was first introduced on 2015 by AWS Lambda, and since then, other commercial and open-source offerings have appeared, i.e., Google Cloud Functions, Azure Functions, Apache OpenWhisk, and others. With FaaS, a developer implements the application logic as stateless functions in a high-level language, such as Java, Python, C, C++, and so on. The code is then packaged together with its dependencies and submitted to the serverless platform. Finally, the developer associates an event to each function, i.e., HTTP requests, file uploads, and more. Once a trigger is fired, the cloud provider executes the code associated with that trigger.

\section{Reinforcement Learning}

Reinforcement Learning considers a problem of a learning agent that actively learns from its own experience. Such agent interacts with its environment and periodically receives a reward signal. The agent’s goal is to maximize the rewards in the long run. However, the agent is not told which actions to take. Instead, it must discover the actions that yield the highest rewards through trial-and-error.

Figure 1 illustrates a typical reinforcement learning scenario. The agent interacts with the environment in discrete time steps. At each time step t, the agent senses the environment’s current state st E S, where S represent the full set of environment states. It then chooses an action at E A(st), where A(st) represents the set of all actions available in the current state. The environment moves to a new state st+1, and the agent receives a reward rt associated with the transition (st,at,st+1).

At any given time, the agent’s behavior is defined by a policy. Roughly speaking, a policy is a mapping from perceived states of the environment to actions to be taken when in those states. The agent’s purpose is to learn the optimal, or near-optimal, policy that maximizes total reward it receives in the long run. 

Exploration vs Exploitation tradeoff
One of the challenges that arises in Reinforcement Learning is the choice between exploiting a familiar action known for a reward and exploring unfamiliar actions for unknown rewards, known as the exploration-exploitation tradeoff. The dilemma is that the agent cannot pursue exploration nor exploitation exclusively without failing the task. Instead, it must find a balance between exploration and exploitation. The agent must try a variety of actions to gather enough information and progressively favor for those that appear to be the best.

QLearning
Q-Learning is a model-free reinforcement learning algorithm, where the agent learns which is the best action to take given the current state. The agent assess the quality of an action by means of a quality-function (Q-function) Q(s,a), denoting the expected total discounted reward if the agent takes action a on state s and acts optimally thereafter. Given the Q-function, the agent’s optimal policy is to choose the action that yields the highest reward.

Figure 4 illustrates the Q-Learning algorithm. The Q-function can be implemented using a simple lookup table. At each step, the agent selects an action a, and observes the reward r and the new state st+1. Then, the agent applies one-step Q-learning, given by:
Qlearning formula
Where 0<alpha<1 is the learning rate and determines to what extent new information overrides the old one. The learned Q-function directly approximates the optimal Q-function, independent of the policy being followed.

%The Q-function can be represented using a simple Q-Table. However, when the state-action space is large, i.e., greater than 10\e{6} states, keeping a lookup table becomes intractable. To solve this problem, the agent needs to build an approximation of the true Q-function using a function approximator.
% \section{System Description}

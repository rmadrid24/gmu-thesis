%%
%% GMU LaTeX PhD Dissertation Format Template
%%
%% Developed by:
%%      Daniel O. Awduche and Christopher A. St. Jean
%%      Communications and Networking Lab
%%      Dept. of Electrical and Computer Engineering
%%
%% Usage note are in this file
%% 
%% 04/27/2023 Tammy Stitz Changed to use the same sty file for theses and dissertations
%% 07/22/2023 Fixed vertical spacing between the text and figures/tables
%%
%%**********************************************************************
%% Legal Notice:
%% This code is offered as-is without any warranty either
%% expressed or implied; without even the implied warranty of
%% MERCHANTABILITY or FITNESS FOR A PARTICULAR PURPOSE!
%% User assumes all risk.
%% In no event shall any contributor to this code be liable for any damages
%% or losses, including, but not limited to, incidental, consequential, or
%% any other damages, resulting from the use or misuse of any information
%% contained here.
%%**********************************************************************
%%
%% $Id: GMU_template.tex,v 1.87 2023/05/10 $
%%
\documentclass[11pt]{report}

%Must be at least 3 lines between the text and the floats TS - 2023
\renewcommand{\textfloatsep}{33pt}
\renewcommand{\floatsep}{33pt}
\renewcommand{\intextsep}{33pt}

%%  The file gmuETD.sty is the GMU latex package to typeset a GMU thesis or dissertation
%%  It should be placed in the same directory as your LaTeX files
\usepackage{gmuETD}

%%
%% other packages that need to be loaded
%%
\usepackage{graphicx}                    %   for imported graphics
\usepackage{amsmath}                     %%
\usepackage{amsfonts}                    %%  for AMS mathematics
\usepackage{amssymb}                     %%
\usepackage{amsthm}                      %%
\usepackage[normalem]{ulem}              %   a nice standard underline package
\usepackage[noadjust,verbose,sort]{cite} %   arranges reference citations neatly
\usepackage{setspace}                    %   for line spacing commands
\usepackage[hyphens]{url}
\usepackage{makecell}
\usepackage[table]{xcolor}
\usepackage{adjustbox}
\usepackage[linesnumbered,ruled,vlined]{algorithm2e}
\usepackage{amsmath}
\usepackage{textcomp}
\usepackage{float}
\usepackage{placeins}

\DeclareMathOperator*{\argmax}{arg\,max}

\beforedoc

\begin{document}

%% In this section, all of the user-specific fields to be used in the
%% title pages are set
%% Note: Title must be in title case to look correct for the title page (e.g., important words are capitalized)
\title{Optimizing Intel Optane DC Persistent Memory Performance\\
            for Serverless Storage}
\onelinetitle{Optimizing Intel Optane DC Persistent Memory Performance for Serverless Storage}
\author{Rafael Alejandro Madrid Rivera}
\credential{MS}
\degree{Master of Science}
\doctype{Thesis}
\dept{Computer Science}
\discipline{Computer Science}

\firstdeg{Bachelor of Science}
\firstdegschool{Central American Technological University}
\firstdegyear{2013}

\degreeyear{2024}

% Note: semester name should be written as Fall Semester, Spring Semester, or Summer Semester.
\degreesemester{Spring Semester}

%Enter all information that will appear below the signature line % (e.g., Dr. Dimitrios Ioannou, Advisor) Need a second line? use \addsigline (e.g., Dr. Firstname Dean, Dean \addsigline of Some School)
% The advisor's name is used in two places. \advisorsignline is on the signature line only. This is needed for the label and if a second line needs added for the signature line
\advisorname{Dr. Yue Cheng}
\makeatletter
\advisorsignline{\@advisorname, \@doctype\ Director}
\makeatother
           
\firstmember{Dr. Fei Li, Committee Member}
\secondmember{Dr. Hakan Aydin, Committee Member}
\depthead{Dr. David Rosenblum, Department Chair}

%%
%% Introductory pages
%%

% Note: The signature sheet is set according to the requirements of the Volgenau School of
% Information Technology and Engineering. If your college/school requirement is different,
% please make appropriate changes in the "signaturepage" section of gmudissertation.sty file.
\signaturepage

\titlepage

% copyright technically optional but should be included in to avoid potential pagination problems
\copyrightpage

%%
%% Dedication page
%%

\dedicationpage

\noindent To God, the source of all wisdom and guidance, I offer my deepest gratitude for providing me with strength and inspiration throughout this academic pursuit. To my cherished parents, whose boundless love, sacrifice, and encouragement have shaped me into the person I am today, I dedicate this thesis with immense appreciation and admiration. And to my beloved future wife, whose unwavering support, patience, and understanding have been my constant source of motivation and joy, I dedicate this work with profound love and gratitude.

%%
%% Acknowledgements
%%

\acknowledgementspage

{
\setlength{\parskip}{1em}
% \noindent The completion of this work owes its realization to the invaluable support and assistance of numerous individuals who have played pivotal roles along this journey.
\noindent Foremost, I express my deepest gratitude to my thesis advisor, Dr.~Yue Cheng. His unwavering support and expert guidance have been instrumental in attaining this academic milestone. His mentorship has left an indelible imprint on my personal and professional growth, for which I am profoundly grateful.
% \noindent Furthermore, I extend my sincere appreciation to the exceptional collaborators who have contributed to this thesis. I would like to thank  Dr.~Dong Li, Dr.~Luna Xu, and Jie Liu for working with me and my advisor over the years. I am particularly grateful to Dr. Dong Li for generously providing access to laboratory resources essential for conducting the experiments presented in this thesis.

\noindent Furthermore, I am deeply grateful to the exceptional collaborators whose contributions have been integral to this thesis. My heartfelt thanks go to Dr.~Dong Li, Dr.~Luna Xu, and Jie Liu for their dedicated collaboration with me and my advisor throughout this endeavor. I am especially appreciative of Dr.~Dong Li for generously providing access to crucial resources essential for conducting the experiments presented in this thesis.

\noindent I am also indebted to my esteemed committee members, Dr.~Fei Li and Dr.~Hakan Aydin, for their invaluable feedback and infinite patience throughout this process.

\noindent On a personal note, I extend heartfelt appreciation to my parents and siblings for their unwavering support over the years. To my parents, Oscar and Mirza, your encouragement and unwavering belief in my abilities have been a constant source of strength. I also acknowledge the support of my extended family, both here in the United States and in Honduras.

\noindent Lastly, I am deeply grateful to my future wife, Sarah, for her unwavering support and encouragement throughout this journey. Her presence has been a constant source of inspiration and motivation.

\noindent To each and every individual who has contributed to this endeavor, I offer my heartfelt thanks. Your support has been invaluable, and I am truly grateful for the role you have played in making this achievement possible.
}

%%
%% Table of contents, list of tables, and lists of figures
%%

\tableofcontents

\listoftables

\listoffigures

%%
%% Abstract
%%
\abstractpage

Intel Optane DC Persistent Memory (Optane PMem) presents a promising solution for developing a serverless storage service. Leveraging its unique attributes of persistence, substantial capacity, and memory-like speeds, this innovative technology holds potential to serve as efficient storage media, offering low latency and high throughput for a variety of applications running on serverless platforms. However, the dynamic and unpredictable characteristics inherent in serverless computing workloads pose challenges to the effective utilization of Optane PMem.

This thesis delves into the utilization of Optane PMem as storage media for serverless computing workloads. Through simulations of real-world serverless applications with diverse workload characteristics and performance requirements, we analyze the limitations of Optane PMem and their impact on latency and throughput service-level agreement (SLA) metrics. Our findings reveal that concurrent execution of applications sharing persistent memory leads to performance degradation and unpredictable behavior from Optane PMem. Moreover, these limitations pose contractual challenges for cloud providers, affecting their ability to meet SLAs.

To tackle these challenges, this work introduces the Non-Volatile Memory (NVM) Middleware, an optimization layer designed to enhance the efficient utilization of Optane PMem within a serverless computing environment. Seamlessly integrated with serverless storage services, the NVM Middleware assumes responsibility for Optane PMem optimization, thereby alleviating cloud providers of such burdens. Aligned with the objectives of an efficient serverless storage service, the NVM Middleware controls the concurrency level on Optane PMem, ensuring performance isolation among co-located applications. Furthermore, leveraging Reinforcement Learning (RL), an agent is trained to dynamically adjust the concurrency level applied by the NVM Middleware in response to changing workloads, enabling cloud providers to meet performance service level agreements even under shifting workloads.

Our evaluation demonstrates that the concurrency control mechanism implemented by the NVM Middleware enhances both latency and throughput exhibited by Optane PMem compared to scenarios lacking concurrency control, resulting in more predictable performance overall. Additionally, our findings indicate that employing an RL agent to dynamically tune resources within the NVM Middleware in response to changing workloads surpasses static resource allocation strategies.

%%
%%  the main body of the dissertation
%%
\startofchapters

%% include the chapter files

%% This file represents a sample first chapter of the main body of the dissertation
%%
%%**********************************************************************
%% Legal Notice:
%% This code is offered as-is without any warranty either
%% expressed or implied; without even the implied warranty of
%% MERCHANTABILITY or FITNESS FOR A PARTICULAR PURPOSE!
%% User assumes all risk.
%% In no event shall any contributor to this code be liable for any damages
%% or losses, including, but not limited to, incidental, consequential, or
%% any other damages, resulting from the use or misuse of any information
%% contained here.
%%**********************************************************************
%%
%% $Id: chapterOne.tex,v 1.6 2006/08/24 21:13:45 Owner Exp $
%%

% A first, optional argument in [ ] is the title as displayed in the table of contents
% The second argument is the title as displayed here.  Use \\ as appropriate in
%   this title to get desired line breaks
\chapter[Introduction]{Introduction}

\section{Motivation}

Serverless computing has experienced significant growth in recent years, emerging as a prominent paradigm for cloud application development. In serverless architecture, developers package their code into stateless functions and deploy it on the serverless platform without the responsibility of managing underlying infrastructure. The serverless platform dynamically scales resources to execute these tasks, billing users only for the resources consumed during function invocations. These characteristics make serverless computing attractive for a diverse range of applications, including web and IoT microservices \cite{gan2019opensource}, video processing \cite{fouladi2017encoding}, data analytics \cite{gimenez2019framework,carver2020wukong,klimovic2018pocket}, machine learning \cite{carreira2019cirrus,feng2018exploring}, and storage applications \cite{10.14778/3587136.3587139,jonas2019cloud}.

Statelessness is a fundamental aspect of serverless functions, as they do not retain state between invocations \cite{jonas2019cloud}. This intentional design choice contributes to high elasticity by eliminating the need to store state within function instances. It enables cloud providers to execute functions in parallel, thereby optimizing resource utilization. Any required data between function invocations must be stored remotely, adhering to the stateless nature of serverless architecture.

While the stateless nature of serverless computing is crucial for achieving high elasticity, it imposes limitations on the types of applications that can efficiently operate on serverless platforms. Previous studies \cite{jonas2019cloud,pu2019shuffling,gan2019opensource} have identified that data-intensive applications, including data analytics, machine learning workflows, and databases, face challenges due to capacity and performance gaps in existing storage services. For instance, object storage services like AWS S3 offer cost-effective long-term storage but suffer from high access latencies. Conversely, in-memory clusters like AWS ElastiCache boast low access latencies and high throughput but are costly and lack transparent provisioning. Meanwhile, key-value databases such as AWS DynamoDB deliver high throughput but are expensive and slow to scale.

Given the constraints of current storage solutions, prior research emphasizes the necessity for a serverless storage service capable of accommodating a diverse array of workloads on serverless platforms. These studies underscore three essential requirements for such a service. Firstly, it must offer low latency and high throughput for various object sizes and data access patterns. Secondly, it should feature transparent provisioning and scalability to accommodate workload fluctuations. Finally, it is imperative that the serverless storage service guarantees isolation and predictable performance across applications and tenants to align with the requirements outlined in service level agreements (SLAs) \cite{jonas2019cloud,klimovic2018pocket}.

To address the requirements outlined in prior research for a serverless storage service, this study focuses on leveraging Intel Optane DC Persistent Memory (PMem) technology. Optane PMem represents a significant advancement in non-volatile memory, offering a blend of persistence and high performance \cite{IntelOp15:online}. Installed on the memory bus alongside traditional DRAM, Optane PMem provides a byte-addressable interface and achieves speeds comparable to DRAM, albeit slightly slower. However, its distinguishing feature lies in its higher capacities and ability to retain data during system shutdowns or power loss, effectively functioning as persistent storage with memory-like speeds \cite{yang2020empirical,izraelevitz2019basic}.

This unique combination of characteristics positions Optane PMem as an ideal solution for accelerating data-intensive workloads in serverless platforms. Therefore, this thesis analyzes the efficient utilization of Optane PMem for developing a serverless storage service that meets the requirements of low latency, high throughput, transparent provisioning, scalability, and performance predictability across applications and tenants, as outlined in service level agreements (SLAs).

% Intel Optane DC PMM differs from DRAM in two ways. First, there is a mismatch between the CPU cacheline access granularity (64-byte) and the 3D-XPoint media access granularity (256-byte) in Intel Optane DC PMM. This difference can lead to write or read amplification if the data access is smaller than 256 bytes. Second, to balance the gap in access granularity, the Intel Optane DC PMM implements a small (16KB) write-combining buffer to merge small writes and reduce write amplification. However, the buffer’s limited capacity (16 KB) can cause contention within the device, limiting its ability to handle access from multiple threads simultaneously.

\section{Research Questions}

With the introduction of Intel Optane PMem, researchers have delved into understanding its characteristics, capabilities, and limitations \cite{izraelevitz2019basic, yang2020empirical, wu2020ribbon}. Initial expectations presumed that Intel Optane DC PMM would behave similarly to DRAM but with lower performance (higher latency and lower bandwidth). However, recent studies suggest that it should not be regarded simply as a “slower, persistent DRAM”. Unlike DRAM, Optane DC PMM displays complex behaviors, and its performance varies based on factors such as access size, access type (read vs. write), and concurrency levels.

One significant difference between Optane PMem and DRAM is that Optane PMem's performance does not scale with higher thread count. This disparity stems from the mismatch between the CPU cacheline access granularity (64-byte) and the 3D-XPoint media access granularity (256-byte) in Optane PMem DIMMs. To address this discrepancy, Intel PMem DIMMs implement a small write-combining buffer to merge small writes and reduce write amplification. However, the buffer’s limited capacity (16 KB) can lead to contention within the device, restricting its ability to handle access from multiple threads simultaneously.

The complex behavior of Intel Optane DC PMM introduces intriguing challenges for constructing a serverless storage service using this technology. Firstly, serverless platforms typically share resources among tenants to execute their applications. These applications vary considerably in multiple aspects, including data access and processing methods, and their performance demands. Secondly, these workloads can experience significant spikes in demand and undergo dramatic changes over time. Understanding how these large-scale variations impact Optane PMem performance can aid in developing an efficient serverless storage service based on this technology.

Given the wide heterogeneity of applications running on serverless platforms, we focus our work on two main types of applications: interactive and batch applications. Interactive applications, such as web-based platforms, facilitate real-time interactions between users and applications. Low latency is crucial to ensure prompt processing of user input and delivery of real-time feedback. Conversely, batch applications, such as data analytics jobs, prioritize high throughput to efficiently process large volumes of data.

Consequently, this thesis addresses the following research questions:

\begin{itemize}
    \item What is the performance degradation (in terms of latency and throughput) observed with Optane PMem when interactive and batch applications concurrently share the storage media? How does this performance degradation impact the service level agreements of serverless applications?
    \item How can we address the limitations of Optane PMem to maximize its efficiency when used as storage media for serverless workloads?
    \item How can we optimize and fine-tune Intel Optane PMem to ensure consistent adherence to service level agreements as workload dynamics evolve over time?
\end{itemize}

\section{Contributions}

Our experiments offer valuable insights into the behavior of Optane PMem when employed as persistent storage for serverless workloads. Firstly, we observe that sharing Optane PMem among multi-threaded interactive and batch applications sim results in performance degradation, characterized by increased latency and reduced bandwidth. This outcome aligns with expectations, considering the contention issues experienced by Optane PMM with higher thread counts. Secondly, we find that the performance degradation in Optane PMM varies depending on the workload, affecting either latency or bandwidth more prominently. This discrepancy implies that certain applications may experience greater QoS impacts than others.

To address the limitations of Intel Optane PMM, we introduce a control layer designed to optimize its utilization within serverless environments. Named the NVM Middleware, this layer integrates with storage services to manage Optane PMM access effectively, mitigating contention. The NVM Middleware implements distinct concurrency levels for interactive and batch applications, ensuring performance isolation across different application types. Leveraging machine learning techniques, the NVM Middleware dynamically adjusts these concurrency levels to meet current demand and adapt to changing workloads. We advocate for the use of online reinforcement learning algorithms, given the evolving data access patterns typical of serverless workloads.

Our contributions can be summarized as follows:

\begin{itemize}
    \item We conduct an experimental study that assesses the capabilities and limitations of Intel Optane PMem as persistent storage for serverless workloads, a scenario not previously explored.
    \item We introduce the NVM Middleware, a control layer designed to reduce contention within Optane PMem while maintaining performance isolation among diverse application types.
    \item We propose a Reinforcement Learning model to train an agent capable of dynamically scaling and provisioning resources within the NVM Middleware to meet service level agreements amidst changing workloads.
    \item Finally, we present empirical evidence demonstrating the effectiveness of our proposed solution.
\end{itemize}

\section{Structure}

The structure of this research paper is outlined as follows:

Chapter 2 provides background information on Intel Optane DC PMem, serverless computing, and Reinforcement Learning (RL). It elaborates on the techniques employed for implementing the RL model within the NVM Middleware.

Chapter 3 details the work carried out on the NVM Middleware. Beginning with a discussion on the essential characteristics of proper serverless storage, the chapter describes how the NVM Middleware contributes to achieving these characteristics while ensuring efficient utilization of Intel Optane DC PMem. It includes an overview of the architecture and programming interface of the NVM Middleware, followed by insights into the Reinforcement Learning model and the Q-Learning algorithm designed to train the NVM Middleware to dynamically adjust concurrency levels under shifting workloads.

Chapter 4 presents an evaluation of the NVM Middleware, encompassing an assessment of its concurrency control mechanism and the performance of the Q-Learning algorithm.

Chapter 5 explores related work in the fields of Intel Optane DC Persistent Memory, serverless storage, and reinforcement learning for dynamic resource allocation. This chapter also discusses the relationship between our work and previous studies, highlighting similarities and differences.

Chapter 6 discusses limitations inherent in this study and proposes potential avenues for future research and development.

Finally, Chapter 7 concludes the research paper by summarizing the presented work and its implications.

%% This file represents a sample second chapter of the main body of the dissertation
%%
%%**********************************************************************
%% Legal Notice:
%% This code is offered as-is without any warranty either
%% expressed or implied; without even the implied warranty of
%% MERCHANTABILITY or FITNESS FOR A PARTICULAR PURPOSE!
%% User assumes all risk.
%% In no event shall any contributor to this code be liable for any damages
%% or losses, including, but not limited to, incidental, consequential, or
%% any other damages, resulting from the use or misuse of any information
%% contained here.
%%**********************************************************************
%%
%% $Id: chapterTwo.tex,v 1.4 2006/08/24 21:12:59 Owner Exp $
%%


% A first, optional argument in [ ] is the title as displayed in the table of contents
% The second argument is the title as displayed here.  Use \\ as appropriate in
%   this title to get desired line breaks
\chapter[Background and Methodology]{Background and Methodology}

\section{Intel Optane DC Persistent Memory}

Persistent memory, also known as Non-volatile Memory (NVM), is a new addition to the memory/storage hierarchy shown in Figure 2 that fills the performance/capacity gap between DRAM and storage by combining traits of both worlds. Like DRAM, persistent memory comes in the form of Dual In-line Memory Modules (DIMMs) that reside on the memory bus. Therefore, applications can access persistent memory like they do with traditional DRAM, eliminating the need to page blocks of data back and forth between memory and storage. However, unlike DRAM DIMMs, persistent memory DIMMs offer greater capacity and can retain data when the system is shutdown or loses power. Thus, persistent memory can dramatically increase system performance and enable a fundamental change in computing architecture.

Intel Optane DC Persistent Memory Module (PMM) is the first commercially available persistent memory technology. This technology comes in DIMM form factor and embeds capacities up to 512GiB. Intel Cascade Lake processors are the first CPUs to support Intel Optane PMM. Like traditional DRAM, the Optane DIMM sits on the memory bus and connects to the processor's integrated memory controller (iMC). Figure 1 shows a typical system configuration of a hybrid node with DRAM and PMM. A user can have up to one Intel Optane DIMM per channel and up to six on a single socket providing capacities up to 3TiB per socket. Thus, an 8-socket system could access up to 24TB of persistent memory.

To ensure persistence, Intel Optane PMM sits within Intel’s asynchronous DRAM refresh (ADR) domain. Intel’s ADR domain ensures that CPU stores that reach the ADR domain will survive a power failure. The iMC maintains read and write pending queues (RPQs and WPQs) for each Optane DIMM and the ADR domain includes WPQs. Once the data reaches the WPQs, the ADR domain ensures that the iMC will flush the updates to persistent memory media on power failure.

The iMC communicates with the Optane DIMM using the DDR-T protocol in cache line access granularity (64B) (Figure 2). The memory access to NVDIMM arrive first at an Apache Pass Controller which coordinates access to the Optane Media. Similar to SSDs, the Optane DIMM perforsms address translation for wear-leveling and bad block management. Thus, it keeps an address indirection table (AIT) for this translation. 

The actual access to storage media occurs after address translation. Intel Optane DIMM physical media access granularity is 256 bytes. Thus, the Controller translates smaller requests into largest 256-byte accesses, causing write amplification as small stores become read-modify-write operations. The controller has a small write-combining buffer to merge adjacent writes.

Intel Optane PMem can operate in two modes: memory and App Direct. Memory mode uses Optane PMem as a large capacity main memory without persistence. DRAM is not visible to the users, and instead it serves as a cache for Optane PMem that is transparently managed by the operating system. In App Direct mode, Optane PMem DIMMs appear as independent, non-volatile storage devices. This allows Optane PMem to be used as a byte-addressable persistent memory that is mapped into the system physical address space and directly accessible by applications [].

% \section{Persistent Memory Development Kit}
% The Persistent Memory Development Kit (PMDK) is a collection of open-source libraries and tools that simplify managing and accessing persistent memory devices. Tuned for both Linux and Windows operating systems, these libraries build on the dax feature described on the SNIA NVM programming specification. The diagram on figure 5 describes the collection of libraries provided by PMDK. Although PMDK’s core libraries provide C APIs, higher level libraries such as pmemkv provide support for other programming systems.

\section{Serverless Computing}

Serverless computing is a cloud computing execution mode that enables developers to deploy their code without provisioning or managing server infrastructure. The term “serverless” is misleading, as servers are still being used by cloud providers to run the code for developers. However, instead of requesting and managing resources, developers simply provide their code, and the cloud providers handle the servers on behalf of their customers. Cloud providers are responsible for provisioning resources, scaling, fault tolerance, monitoring, security patches, and so on. Finally, developers simply pay by the execution time and resources used on their code invocations.

Function-as-a-service (FaaS) is the core compute engine for serverless computing. It was first introduced on 2015 by AWS Lambda, and since then, other commercial and open-source offerings have appeared, i.e., Google Cloud Functions, Azure Functions, Apache OpenWhisk, and others. With FaaS, a developer implements the application logic as stateless functions in a high-level language, such as Java, Python, C, C++, and so on. The code is then packaged together with its dependencies and submitted to the serverless platform. Finally, the developer associates an event to each function, i.e., HTTP requests, file uploads, and more. Once a trigger is fired, the cloud provider executes the code associated with that trigger.

\section{Reinforcement Learning}

Reinforcement Learning (RL) is a subfield of machine learning concerned with learning optimal decision-making policies through interactions with an environment \cite{sutton2018reinforcement}. The fundamental concept underlying RL is the notion of an agent, which takes actions in an environment and receives feedback in the form of rewards, indicating the quality of its decisions. The agent's objective is to learn a policy that maximizes cumulative rewards over time. Moreover, the agent is not provided with explicit instructions on which actions to take; instead, it must discover the actions that lead to the highest rewards by trying them.

\begin{figure}[ht]
    \centering
    \includegraphics[scale=1]{images/rl-workflow.png}
    \caption{RL Workflow}
    \label{fig:sutton_rl_workflow}
\end{figure}

Figure \ref{fig:sutton_rl_workflow} presents a schematic representation of a standard reinforcement learning scenario. In discrete time steps, the agent perceives the current state $s_t$ from the set of all possible states $S$. It then selects an action $a_t$ from the available actions $A(s_t)$ in the current state. The environment transitions to a new state $s_{t+1}$, and the agent receives a reward $r_t$ associated with the transition $(s_t, a_t, s_{t+1})$.

The agent's behavior is governed by its policy, which maps perceived states to actions. The ultimate aim is to learn an optimal or near-optimal policy that maximizes the cumulative reward. 


\subsubsection*{Q-Learning}

One of the foundational algorithms in RL is Q-Learning, introduced by Watkins in 1989 \cite{watkins1989learning}. The algorithm belongs to the class of model-free RL algorithms, meaning it learns directly from experience without requiring a model of the environment dynamics \cite{russel2020ai}.

At the core of Q-Learning is the Q-value function, denoted as $Q(s, a)$, which represents the expected cumulative reward the agent will receive by taking action $a$ in state $s$ and following an optimal policy thereafter. The objective of Q-Learning is to iteratively update the Q-values based on observed transitions and rewards, eventually converging to the optimal Q-values that maximize long-term rewards.

The Q-Learning algorithm proceeds as follows: the agent interacts with the environment by selecting actions based on its current estimate of the Q-values. Upon taking an action, the agent observes the resulting reward and the next state. It then updates the Q-value of the previous state-action pair using the observed reward and the estimated value of the next state.

The Q-value update rule in Q-Learning is based on the Bellman equation, which expresses the relationship between the Q-values of successive states \cite{russel2020ai}:

\[
Q(s, a) \leftarrow (1 - \alpha) \cdot Q(s, a) + \alpha \cdot \left( r + \gamma \cdot \max_{a'} Q(s', a') \right)
\]

Here, $\alpha$ is the learning rate, determining the extent to which new information overrides the old one, and $\gamma$ is the discount factor, representing the importance of future rewards relative to immediate rewards. The term $r + \gamma \cdot \max_{a'} Q(s', a')$ is known as the temporal-difference (TD) target, combining the immediate reward $r$ with the discounted maximum Q-value of the next state $s'$ \cite{russel2020ai}.

% Q-Learning, a model-free reinforcement learning algorithm, is employed by the agent to determine the best action given the current state. The agent evaluates action quality using a quality-function (Q-function) $Q(s, a)$, representing the expected total discounted reward if the agent selects action $a$ in state $s$ and acts optimally thereafter. 

% One of the foundational algorithms in RL is Q-Learning, introduced by Watkins in 1989 \cite{watkins1989learning}. Q-Learning is a model-free algorithm that learns the value of taking an action in a particular state, known as the Q-value, and iteratively refines these values through experience. The Q-value represents the expected cumulative reward the agent will receive by taking an action in a given state and following an optimal policy thereafter.

% The Q-Learning algorithm (illustrated in Figure \ref{algo:q_learning}) involves iterative updates to the Q-function. At each step, the agent selects an action, observes the reward and new state, and then applies one-step Q-learning. The update is governed by the Q-learning formula, where the learning rate ($\alpha$) determines the extent to which new information overrides old data. This learned Q-function approximates the optimal Q-function, irrespective of the policy being followed.

\subsubsection{Linear Regression Models in Reinforcement Learning}

One of the key advantages of Q-Learning is its simplicity and ease of implementation. It requires only a table to store the Q-values, making it computationally efficient for small state and action spaces. However, Q-Learning faces challenges in environments with large state spaces, as maintaining a lookup table becomes infeasible due to memory and computational constraints.

Function approximation is a fundamental technique in reinforcement learning (RL) aimed at approximating the Q-Value function when dealing with large state or action spaces where tabular representations become impractical \cite{russel2020ai}. This approach allows RL agents to generalize from observed states to unseen states, facilitating decision-making in unexplored regions of the state space.

In the context of RL, linear regression models are commonly used for function approximation \cite{sutton2018reinforcement}.  These models approximate the Q-value function by leveraging a weighted linear combination of features, with each feature capturing a distinct aspect of the state space. Employing gradient-descent methods, notably stochastic gradient descent, enables iterative refinement of the parameters governing the linear function, aimed at minimizing a predefined loss function. This iterative optimization process empowers the model to progressively enhance its predictive accuracy and capture intricate patterns within the state-action space.

Hyperparameter tuning is a critical aspect of training linear regression models in RL \cite{bergstra2012random}. Hyperparameters, such as the learning rate, regularization strength, and feature scaling, significantly impact the performance and convergence of the models. A systematic approach to hyperparameter tuning involves experimenting with different combinations of hyperparameters, evaluating the performance of the trained models on a validation set, and selecting the optimal hyperparameters based on predefined criteria, such as validation error or performance metrics \cite{russel2020ai}.

\subsubsection{Exploration-Exploitation Tradeoff}

The exploration-exploitation tradeoff poses a significant challenge in reinforcement learning \cite{sutton2018reinforcement}. The agent must strike a balance between exploring unfamiliar actions to gather information and exploiting known actions for immediate rewards. Finding this balance is crucial for effective learning and task performance, as the agent gradually favors actions with higher expected rewards.

One classic strategy for balancing exploration and exploitation is the epsilon-greedy (e-greedy) algorithm \cite{sutton2018reinforcement}. The e-greedy policy selects the action that maximizes the estimated value with probability $1 - \epsilon$ (exploitation) and selects a random action with probability $\epsilon$ (exploration). This approach ensures that the agent continues to explore the environment while gradually exploiting more rewarding actions as it gains knowledge.

Decayed e-greedy methods aim to strike a balance between exploration and exploitation by gradually reducing the exploration rate $\epsilon$ as the agent gains more experience or as the training progresses \cite{sutton2018reinforcement}. This decay encourages the agent to explore the environment more extensively in the early stages of learning while gradually shifting towards exploitation as it becomes more knowledgeable.

\subsubsection{Reward shaping}

Reward shaping is a technique in reinforcement learning (RL) aimed at accelerating learning by modifying the reward signal provided to the agent. Traditional RL algorithms rely solely on sparse reward signals, which can make learning slow and inefficient, especially in complex environments. Reward shaping addresses this issue by providing additional, shaped rewards that guide the agent towards desirable behaviors. These shaped rewards are designed to provide more informative feedback to the agent, encouraging it to explore the state-action space more effectively. However, reward shaping must be carefully designed to avoid unintended consequences such as overfitting to the shaped rewards or incentivizing undesirable behaviors \cite{russel2020ai}.

% Reinforcement Learning considers a problem of a learning agent that actively learns from its own experience \cite{sutton2018reinforcement}. Such agent interacts with its environment and periodically receives a reward signal. The agent’s goal is to maximize the rewards in the long run. However, the agent is not told which actions to take. Instead, it must discover the actions that yield the highest rewards through trial-and-error.

% \begin{figure}[ht]
%     \centering
%     \includegraphics[scale=1]{images/rl-workflow.png}
%     \caption[RL Workflow]{RL Workflow}
%     \label{fig:sutton_rl_workflow}
% \end{figure}

% Figure \ref{fig:sutton_rl_workflow} illustrates a typical reinforcement learning scenario. The agent interacts with the environment in discrete time steps. At each time step t, the agent senses the environment’s current state st E S, where S represent the full set of environment states. It then chooses an action at E A(st), where A(st) represents the set of all actions available in the current state. The environment moves to a new state st+1, and the agent receives a reward rt associated with the transition (st,at,st+1).

% At any given time, the agent’s behavior is defined by a policy. Roughly speaking, a policy is a mapping from perceived states of the environment to actions to be taken when in those states. The agent’s purpose is to learn the optimal, or near-optimal, policy that maximizes total reward it receives in the long run. 
% QLearning
% Q-Learning is a model-free reinforcement learning algorithm, where the agent learns which is the best action to take given the current state. The agent assess the quality of an action by means of a quality-function (Q-function) Q(s,a), denoting the expected total discounted reward if the agent takes action a on state s and acts optimally thereafter. Given the Q-function, the agent’s optimal policy is to choose the action that yields the highest reward.

% Figure 4 illustrates the Q-Learning algorithm. The Q-function can be implemented using a simple lookup table. At each step, the agent selects an action a, and observes the reward r and the new state st+1. Then, the agent applies one-step Q-learning, given by:
% Qlearning formula
% Where 0<alpha<1 is the learning rate and determines to what extent new information overrides the old one. The learned Q-function directly approximates the optimal Q-function, independent of the policy being followed.

% The Q-function can be represented using a simple Q-Table. However, when the state-action space is large, i.e., greater than $10^{6}$ states, keeping a lookup table becomes intractable. To solve this problem, the agent needs to build an approximation of the true Q-function using a function approximator.

% Exploration vs Exploitation tradeoff
% One of the challenges that arises in Reinforcement Learning is the choice between exploiting a familiar action known for a reward and exploring unfamiliar actions for unknown rewards, known as the exploration-exploitation tradeoff. The dilemma is that the agent cannot pursue exploration nor exploitation exclusively without failing the task. Instead, it must find a balance between exploration and exploitation. The agent must try a variety of actions to gather enough information and progressively favor for those that appear to be the best.
\chapter[A shim Layer for persistent memory]{A shim layer for persistent memory}

As we have discussed, the release of Intel Optane PMM opens a major opportunity for serverless storage services. This memory technology provides a unique combination of affordable larger capacity, high-performance, and support for data persistence \cite{IntelOp15:online}. When configured in App-Direct mode, the Optane DIMM and DRAM DIMMs act as independent memory resources under direct load/store control of the applications. This allows the Optane PMM capacity to be used as byte-addressable persistent memory that is mapped into the system application space and directly accessible by applications. Together, these advantages enable Optane PMM to be used as persistent storage with memory-like speeds.

Unfortunately, the resource contention observed within Optane PMM can impose serious performance and contractual implications for a multi-tenant serverless storage service. Given the hallmark autoscaling features of serverless computing, the memory’s limited ability to handle accesses from multiple threads can degrade the overall system’s performance when workload spikes occur. Furthermore, these storage systems make efficient use of their infrastructure by allowing multiple users, or tenants, to share the physical resources. The performance degradation caused by Optane PMM can lead tenants to experience significant performance variations. The latter inhibits service providers from offering certain service level agreements.

To reduce the contention effect, previous studies recommend limiting the number of threads that access Optane PMM simultaneously. In \cite{yang2020empirical}, Yang et. al they improve the performance of an NVM-aware file system by limiting the number of writer threads that access each Optane DIMM. Similarly, Ribbon \cite{wu2020ribbon} controls the number of threads performing CLF and adjusts this number dynamically at runtime. While this approach provides a viable solution, it introduces management problems for a system administrator of a multi-tenant serverless storage.

Given the complexity of serverless computing workloads, implementing efficient concurrency control mechanisms for optimizing an Optane-based serverless storage service is a challenging task. These challenges are discussed in section 3.1, but in short, service providers have three crucial tasks when implementing these control mechanisms. First, they must provide predictable performance, ensuring that all the SLAs from different workloads are met. Second, they must scale resources transparently to meet the current workload demand. Finally, they must come up with policies that allow their system to adapt quickly to sudden workload shifts. To this end, we propose a solution that takes on these responsibilities from the service providers.

In this work, we present a shim layer that addresses the shortcomings of Intel Optane PMM highlighted above, while meeting the different service level agreements from multiple tenants under shifting workloads. Our shim layer, called NVM Middleware, distinguishes between latency-critical and throughput-oriented workloads and applies different concurrency control mechanisms for each one. This enables the system to reduce the contention on the memory device, as well as the interference among workloads with different service level agreements. In addition, we propose the development of a reinforcement learning agent to adapt the NVM Middleware quickly to changing workloads. The agent takes into account the characteristics and service level agreements and learns from past experiences to scale resources accordingly.


\section{Motivation}
In this section, we discuss the pain points of controlling the number of threads to optimize Optane PMM within a serverless storage service and explain the design goals of the NVM Middleware.

\subsection{Concurrency Control Challenges in a serverless storage service}

When building an Optane PMM based serverless storage service, optimizing the memory's performance is just the start. Early works in serverless computing have identified several tasks that a storage service must perform efficiently to meet the demands of serverless computing \cite{180275,jonas2019cloud,klimovic2018understanding,klimovic2018pocket,wu2019autoscaling,romero2021faat}. As a result, service providers must ensure that their concurrency control policies do not interfere with these design goals. In this work, we focus on three challenges faced by service providers when designing a high-performance storage service based on Optane PMM.

\textbf{Support for a wide heterogeneity of applications.} In serverless computing, users typically deploy their applications as a collection of serverless functions that share data among them using remote storage. Previous studies suggest that these applications vary considerably in the way store, distribute, and process data. This diversity is reflected in multiple ways, such as data access size \cite{klimovic2018pocket,romero2021faat}, data access patterns \cite{romero2021faat}, and their performance requirements [180275,jonas2019cloud]. Therefore, service providers face the challenge of tuning the concurrency level to support many types of applications. In this work, we argue that considering the workload characteristics is key for tuning the system efficiently. The allocation of resources can vary depending on the workload type.

\textbf{Compliance with Service Level Agreements.} The success of a storage service relies on its ability to comply with various service level agreements (SLAs). SLAs play a critical role in governing the relationship between the storage provider and its customers. They help establish clear expectations between both parties regarding the quality of storage service. Therefore, service providers face the challenge of staying in compliance with these SLAs while they seek to optimize Optane PMM. 

\textbf{Automatic and transparent scaling.} Serverless workloads are extremely unpredictable. These workloads can launch hundreds of functions instantaneously to meet application demands \cite{klimovic2018understanding}. Furthermore, the data access patterns of the applications can change dramatically over time \cite{romero2021faat,wu2019autoscaling}. Service providers face the challenge of scaling the resources, such as number of threads, automatically to meet the demands of changing workloads. In addition, they must ensure that the system adapts quickly enough to avoid missing SLAs.


\subsection{NVM Middleware Design Overview}

We design NVM Middleware with three main design goals.

\textbf{Workload-aware Contention Management.} We focus our work on two main types of workloads: interactive and batch applications. Interactive applications, such as web-based platforms, enable real-time interactions between the user and the application. Low latency is critical to ensure that the user input is processed quickly, and feedback is delivered in real-time. On the other hand, batch applications, such as data analytics jobs, facilitate efficient processing of large-scale data. These workloads prioritize high throughput to process large volumes of data efficiently.

The NVM Middleware leverage insights about the workload characteristics, resource demands, and performance requirements of applications to make informed decisions about resource allocation and contention resolution. By dynamically adjusting resource allocation and contention resolution mechanisms based on the workload characteristics, the NVM Middleware mitigates contention-induced performance degradation and ensures efficient resource sharing among co-located applications. This adaptive approach enables the NVM Middleware to allocate resources judiciously to maximize overall system efficiency and meet diverse performance requirements of both interactive and batch applications. By using the content-aware contention management offered by the NVM Middleware, a storage system using Optane PMM can effectively balance the needs of different workload types, ensuring optimal performance and resources utilizing in multi-tenant environments.

\textbf{SLA-driven autoscaling policies.} The NVM Middleware leverages SLAs, which define the quality-of-service parameters agreed up between the service provider and their customers, to dynamically adjust contention resolution mechanisms in response to changes in service level agreement metrics. It continuously monitors SLA metrics, such as 99th latency and throughput, and evaluates its own performance against predefined SLA targets. This real-time monitoring allows the NVM Middleware to detect deviations from SLA requirements and triggers scaling actions to dynamically adjust resource allocation. By aligning resource provisioning with SLA requirements, the NVM Middleware can ensure a consistent and reliable performance from Optane PMM, even under dynamic workload changes.

\textbf{RL-driven autoscaling policies.} Besides leveraging SLAs to dynamically provision resources and adjust contention resolution mechanisms, our solution proposes the use of Reinforcement Learning to learn from past experiences and predict future behaviors.  These RL-driven policies enable the NVM Middleware to adapt to changing workload patterns over time and meet SLAs objectives more effectively than traditional threshold-based approaches []. Moreover, given the dynamic and unpredictable of serverless workloads, we propose a model-free algorithm, Q-Learning, to continuously learn the optimal policy based on observed experiences, allowing the NVM Middleware to adapt to new scenarios without needing to explicitly model them.

\section{Architecture}

\begin{figure}[ht]
  \centering
  \includegraphics[scale=1]{images/nvm_design.png}
  \caption[NVM Middleware Architecture]{NVM Middleware Architecture}
  \label{fig:nvm_architecture}
\end{figure}

Figure \ref{fig:nvm_architecture} provides an overview of the NVM Middleware architecture. Positioned as a middle layer connecting user applications with Optane PMM, its design is tailored for seamless integration within a storage service, serving as an optimization layer specifically targeting Optane PMM. It comprises a request handler, two concurrency thread pools, and a monitoring and resource management module.

The request handler serves as the primary interface for handling user I/O requests. Upon receipt, it segregates requests into two distinct non-blocking First-In-First-Out (FIFO) queues: one tailed for latency-sensitive requests and the other for throughput-centric ones. Leveraging insights into workload characteristics, the handler intelligently allocates requests to the appropriate queue. Moreover, each queue is assigned a dedicated pool of worker threads tasked with dispatching I/O requests to Optane PMM using PMEMKV. Notably, these thread pools operate independently and are dynamically managed and scaled by the Reinforcement Learning agent to meet predetermined latency and throughput goals.

The Monitoring and Resource Management module offers an interface to monitor system load and SLA performance metrics. This module initiates a separate control thread tasked with gathering data on key parameters within the NVM Middleware, such as 99th latency, throughput, and system load. Utilizing this information, the RL agent makes data-driven decisions regarding optimal thread pool scaling. Subsequently, these decisions are communicated to the Monitoring and Resource Management module, which executes the required actions within the NVM Middleware.

\section{Programming Interface}

% \begin{table}[ht]
%   \centering
%   \caption{Programming Interface}
%   \label{table:programming_interface}
%   % Tabular environment goes AFTER the caption!
%   \begin{adjustbox}{width=1\textwidth}
%   \begin{tabular}{|l|l|l}
%     % after \\: \hline or \cline{col1-col2} \cline{col3-col4} ...
%     \hline
%     \thead{API Name} & \thead{Functionality} \\
%     \hline
%     \rowcolor{gray!50} % Color the header row
%     start(db, interactiveThreads, batchThreads) & \makecell[cl] {Create PMEMKV database. \\ Start control threads. \\ Start system monitoring in Monitoring and Resource Management Module.} \\
%     close() & \makecell[cl] {Close PMEMKV database. \\ Stop all threads. \\ Start system monitoring in Monitoring and Resource Management Module.} \\\hline
%     \rowcolor{gray!50}
%     get(key, mode) & Retrieves key from persistent memory. \\
%     put(key, value, mode) & Writes key to persistent memory. \\
%     \hline
%     \rowcolor{gray!50}
%     get\_stats() & Provides the 99th percentile and throughput observed by the NVM Middleware. \\
%     get\_state() & Provides the current state within the NVM Middleware. \\
%     \rowcolor{gray!50}
%     perform\_action(action) & Triggers a scaling action. \\
%     \hline
%   \end{tabular}
% \end{adjustbox}
% \end{table}

% Table \ref{table:programming_interface} provides an overview of the NVM Middleware application interface, which features a key-value API supplemented with functions tailored to facilitate the decision-making process associated with thread pool scaling.

% The start function initializes the database utilized by PMEMKV, initializes the thread pools with an initial thread count, and triggers the system monitoring within the Monitoring and Resource Management Module. Conversely, the stop function terminates the connection with the PMEMKV database, releases resources utilized by the thread pools, and prompts the Monitoring and Resource Management to cease system monitoring. The API enables Ke-Value Store interactions via get and put functions, with the capacity to accept hints concerning the request type (e.g., latency-sensitive or throughput-oriented) to assist the request handler in queue allocation.

% Moreover, the get\_stats function furnishes insights into the 99th percentile and throughput observed by the NVM Middleware at any given moment. Similarly, the get\_state function provides real-time state information as outlined in \ref{table:state_space}. Finally, the perform\_action function accepts scaling actions detailed in \ref{table:action_space} and initiates the corresponding action within the NVM Middleware.

\begin{table}[ht]
  \centering
  \caption{Programming Interface}
  \label{table:programming_interface}
  % Tabular environment goes AFTER the caption!
  \begin{adjustbox}{width=1\textwidth}
  \begin{tabular}{|l|l|l|}
    % after \\: \hline or \cline{col1-col2} \cline{col3-col4} ...
    \hline
    \thead{Category} & \thead{API Name} & \thead{Functionality} \\
    \hline
    \rowcolor{gray!50} % Color the header row
    System & start(db, interactiveThreads, batchThreads) & \makecell[cl]{Create PMEMKV database. \\ Start interactive and batch thread pools. \\ Initiate system monitoring in Monitoring and Resource Management Module.} \\
    System & stop() & \makecell[cl]{Closes PMEMKV database. \\ Stop thread pools. \\ Stop system monitoring.} \\
    \hline
    \rowcolor{gray!50}
    System & get(key, mode) & Retrieves key from persistent memory. \\
    System & put(key, value, mode) & Writes key to persistent memory. \\
    \rowcolor{gray!50}
    RL & get\_stats() & Provides the 99th percentile and throughput observed by the NVM Middleware. \\
    RL & get\_state() & Provides the current state within the NVM Middleware. \\
    \rowcolor{gray!50}
    RL & perform\_action(action) & Triggers a scaling action. \\
    \hline
  \end{tabular}
\end{adjustbox}
\end{table}


Table \ref{table:programming_interface} outlines the NVM Middleware's programming interface, presenting a set of functions designed to facilitate interaction with a storage system and the reinforcement learning agent. 

The $start$ function initializes the PMEMKV database, initializes the thread pools with an initial thread count, and triggers the system monitoring within the Monitoring and Resource Management Module. In contrast, the $stop$ function terminates the database connection, halts all threads in the thread pools, and stops system monitoring. Furthermore, the $get$ and $put$ functions facilitate key-value interactions with the persistent memory, allowing for read and write operations. These functions are designed to accommodate hints regarding the request type (e.g., latency-sensitive or throughput-oriented), aiding the request handler in queue allocation.

The $get\_stats$ function furnishes insights into the 99th percentile and throughput observed by the NVM Middleware at any given moment. Similarly, the $get\_state$ function provides real-time state information as outlined in Table \ref{table:state_space}. Finally, the $perform\_action$ function accepts scaling actions detailed in Table \ref{table:action_space} and initiates the corresponding action within the NVM Middleware.

\section{Reinforcement Learning Component}

In this section, we discuss the Q-learning algorithm used by the Reinforcement Learning agent to dynamically adjust the number of threads assigned to each thread pool. The agent’s goal is to find the best combination of threads that meets predetermined latency and throughput SLAs while minimizing contention and ensuring efficient utilization of Intel Optane PMM. 

\subsection{Integration with the NVM Middleware}

\begin{figure}[ht]
  \centering
  \includegraphics[scale=1]{images/rl_workflow.png}
  \caption[RL Workflow]{RL Workflow}
  \label{fig:rl_workflow}
\end{figure}

Figure \ref{fig:rl_workflow} offers a visual representation of the interaction between the reinforcement learning (RL) agent and the NVM Middleware. At each time step, the NVM Middleware receives a diverse influx of requests, spanning both latency-sensitive and throughput-oriented tasks. These requests necessitate translation into actionable I/O commands directed towards the Intel Optane Persistent Memory Module (PMM).

Concurrently, the RL agent adeptly captures the environment's current state, leveraging real-time workloads' characteristics and performance metrics provided by the monitoring module. Utilizing this information, the agent orchestrates the selection of an optimal action, guiding the dynamic adjustment of threads within the interactive and batch thread pools. This adaptive decision-making process is exemplified by actions like augmenting the count of interactive threads to address evolving workload demands.

Following action selection, the NVM Middleware's resource management module implements the chosen course of action, fine-tuning the NVM Middleware's interactive and batch threads to efficiently handle incoming user requests. Upon the completion of each time step, the action's effectiveness is rigorously assessed against predefined service level agreement (SLA) targets, yielding a reward signal generated by a reward manager.

This reward serves as invaluable feedback for the RL agent, empowering iterative policy updates aimed at refining decision-making strategies in subsequent time steps. Thus, the presented framework embodies a recursive learning cycle, wherein the RL agent continuously hones its behavior through real-world interactions, ensuring adaptive responsiveness to evolving workload dynamics.

\subsection{Reinforcement Learning Model}

\subsubsection{State Space}
% Table \ref{table:state_space} provides an overview of the features used in our state representation. We define state s at time step t by the the following tuple st= (interactiveThreads, batchThreads, InteractiveQueueSize, batchQueueSize, interactiveBlockSize, batchBlockSize, interactiveRWRatio, batchRWRatio), where st E S.

% \begin{table}[ht]
%   \centering
%   \caption{The State Representation}
%   \label{table:state_space}
%   \begin{adjustbox}{width=1\textwidth}
%   \begin{tabular}{|l|l|l|}
%     \hline
%     \thead{Name} & \thead{Description} & \thead{Values} \\
%     \hline
%     \rowcolor{gray!50}
%     interactiveThreads & Number of (interactive) threads assigned to the interactive thread pool. & $1 \leq \text{interactiveThreads} \leq 32$ \\
%     \hline
%     batchThreads & Number of (batch) threads assigned to the batch thread pool. & $1 \leq \text{batchThreads} \leq 32$ \\
%     \hline
%     \rowcolor{gray!50}
%     interactiveQueueSize & Number of requests in the interactive queue. & $\text{interactiveQueueSize} \in \mathbb{R}^+$ \\
%     \hline
%     batchQueueSize & Number of requests in the batch queue. & $\text{batchQueueSize} \in \mathbb{R}^+$ \\
%     \hline
%     \rowcolor{gray!50}
%     interactiveBlockSize & Average block size of interactive workload. & $\text{interactiveBlockSize} \in \mathbb{R}^+$ \\
%     \hline
%     batchBlockSize & Average block size of batch workload. & $\text{batchBlockSize} \in \mathbb{R}^+$ \\
%     \hline
%     \rowcolor{gray!50}
%     interactiveWriteRatio & Proportion of write requests compared to read requests in the interactive workload. & $\text{interactiveRWRatio} \in \mathbb{R}^+$ \\
%     \hline
%     batchWriteRatio & Proportion of write requests compared to read requests in the batch workload & $\text{batchRWRatio} \in \mathbb{R}^+$ \\
%     \hline
%   \end{tabular}
%   \end{adjustbox}
% \end{table}
\begin{table}[ht]
  \centering
  \caption{The State Representation}
  \label{table:state_space}
  \begin{adjustbox}{width=1\textwidth}
  \begin{tabular}{|l|l|l|}
    \hline
    \thead{Name} & \thead{Description} & \thead{Values} \\
    \hline
    \rowcolor{gray!50}
    interactiveThreads & \makecell[l]{Number of (interactive) threads assigned\\ to the interactive thread pool.} & $1 \leq \text{interactiveThreads} \leq 32$ \\
    \hline
    batchThreads & \makecell[l]{Number of (batch) threads assigned\\ to the batch thread pool.} & $1 \leq \text{batchThreads} \leq 32$ \\
    \hline
    \rowcolor{gray!50}
    interactiveQueueSize & \makecell[l]{Number of requests in the interactive queue.} & $\text{interactiveQueueSize} \in \mathbb{R}^+$ \\
    \hline
    batchQueueSize & \makecell[l]{Number of requests in the batch queue.} & $\text{batchQueueSize} \in \mathbb{R}^+$ \\
    \hline
    \rowcolor{gray!50}
    interactiveBlockSize & \makecell[l]{Average block size of interactive workload.} & $\text{interactiveBlockSize} \in \mathbb{R}^+$ \\
    \hline
    batchBlockSize & \makecell[l]{Average block size of batch workload.} & $\text{batchBlockSize} \in \mathbb{R}^+$ \\
    \hline
    \rowcolor{gray!50}
    interactiveWriteRatio & \makecell[l]{Proportion of write requests compared\\ to read requests in the interactive workload.} & $\text{interactiveRWRatio} \in \mathbb{R}^+$ \\
    \hline
    batchWriteRatio & \makecell[l]{Proportion of write requests compared\\ to read requests in the batch workload.} & $\text{batchRWRatio} \in \mathbb{R}^+$ \\
    \hline
  \end{tabular}
  \end{adjustbox}
\end{table}

Table \ref{table:state_space} presents the features of our state representation. At each time step $t$, we define the state $s_t$ as a tuple:

\[
\begin{aligned}
s_t = (& \text{interactiveThreads}_t, \text{batchThreads}_t, \text{InteractiveQueueSize}_t, \text{batchQueueSize}_t, \\
& \text{interactiveBlockSize}_t, \text{batchBlockSize}_t, \text{interactiveRWRatio}_t, \text{batchRWRatio}_t )
\end{aligned}
\]

where $s_t \in S$ represents the state space. The tuple encapsulates the key features characterizing the system's current state, including the number of interactive and batch threads, number of pending requests in the queues, individual workload block sizes, and write ratio for both interactive and batch workloads.

\subsubsection{Action Space}

\begin{table}[ht]
  \centering
  \caption{Possible Actions in the Action Space}
  \label{table:action_space}
  \begin{tabular}{|c|l|l|}
  \hline
  \thead{Action} & \thead{Effect on \ Interactive Threads} & \thead{Effect on \ Batch Threads} \\
  \hline
  0 & No change & No change \\
  1 & Increase by 1 & No change \\
  2 & Decrease by 1 & No change \\
  3 & No change & Increase by 1 \\
  4 & No change & Decrease by 1 \\
  5 & Increase by 1 & Increase by 1 \\
  6 & Increase by 1 & Decrease by 1 \\
  7 & Decrease by 1 & Increase by 1 \\
  8 & Decrease by 1 & Decrease by 1 \\
  \hline
  \end{tabular}
  \end{table}

  Table \ref{table:action_space} illustrates the feasible actions within the action space. Each action corresponds to a potential adjustment in the number of interactive and batch threads. The table enumerates nine distinct actions, each with its respective effect on the number of interactive threads and batch threads.

  Mathematically, the set of actions $A$ is defined as $A = \{0,1,2,3,4,5,6,7,8\}$ for a given state $s_t \in S$.

\subsubsection{Reward}

\begin{algorithm}[ht]
  % \small
  \caption{Reward Calculation Algorithm}
  \label{algo:reward_calculation}
  \SetAlgoLined
  \KwIn{System statistics: $\text{stat}$}
  \KwOut{Reward value: $\text{reward}$}
  \tcc{Initialize variables}
  $\text{max\_scale\_lat} \leftarrow 1000$, $\text{max\_scale\_tp} \leftarrow 10$, $\text{min\_scale} \leftarrow 1$, $\text{lat\_goal} \leftarrow 200$, $\text{tp\_goal} \leftarrow 250000$, $\text{lat\_penalty} \leftarrow 500.0$, $\text{tp\_penalty} \leftarrow 5000.0$\;
    
  \tcc{Scale observed and target latency and throughput}
  $\text{lat} \leftarrow ((\text{max\_scale\_lat} - \text{min\_scale}) \times (\text{stat.tailLatency} - \text{min\_value}) / (\text{max\_latency} - \text{min\_value})) + \text{min\_scale}$\;
  $\text{tp} \leftarrow ((\text{max\_scale\_tp} - \text{min\_scale}) \times (\text{stat.throughput} - \text{min\_value}) / (\text{max\_throughput} - \text{min\_value})) + \text{min\_scale}$\;
  $\text{lat\_goal} \leftarrow ((\text{max\_scale\_lat} - \text{min\_scale}) \times (\text{lat\_goal} - \text{min\_value}) / (\text{max\_latency} - \text{min\_value})) + \text{min\_scale}$\;
  $\text{tp\_goal} \leftarrow ((\text{max\_scale\_tp} - \text{min\_scale}) \times (\text{tp\_goal} - \text{min\_value}) / (\text{max\_throughput} - \text{min\_value})) + \text{min\_scale}$\;
    
  \tcc{Calculate errors}
  $\text{error\_lat} \leftarrow |\text{lat} - \text{lat\_goal}|$\;
  $\text{error\_tp} \leftarrow |\text{tp} - \text{tp\_goal}|$\;
    
  \tcc{Calculate reward}
  \eIf{$\text{lat} \leq \text{lat\_goal}$ \textbf{and} $\text{tp} \geq \text{tp\_goal}$}{
      $\text{reward} \leftarrow 10 \times (\text{error\_lat} + \text{error\_tp}$) \tcp*{High reward for meeting both latency and throughput goals}
  }{
      \eIf{$\text{lat} > \text{lat\_goal}$ \textbf{and} $\text{tp} < \text{tp\_goal}$}{
          $\text{reward} \leftarrow -1 \times (\text{lat\_penalty} \times \text{error\_lat} + \text{tp\_penalty} \times \text{error\_tp})$ \tcp*{Penalize for high latency and low throughput}
      }{
          \eIf{$\text{lat} > \text{lat\_goal}$}{
              $\text{reward} \leftarrow -1 \times \text{lat\_penalty} \times \text{error\_lat}$ \tcp*{Penalize for high latency}
          }{
              $\text{reward} \leftarrow -1 \times \text{tp\_penalty} \times \text{error\_tp}$ \tcp*{Penalize for low throughput}
          }
      }
  }

\end{algorithm}

To guide the optimization process of the reinforcement learning agent, we establish an algorithm (Algorithm \ref{algo:reward_calculation}) to calculate a reward value based on observed and target latency and throughput metrics. This algorithm, outlined below, serves as a crucial component in training the RL agent to make informed decisions.

\begin{enumerate}
  \item Lines 1-5 define goals, scaling factors, and penalties. The observed and target latency ($lat$, $lat\_goal$) and throughput ($tp$, $tp\_goal$) metrics are scaled to a normalized range using scaling factors ($max\_scale\_lat$, $max\_scale\_tp$) and minimum scale ($min\_scale$). This normalization process ensures that both metrics contribute proportionally to the reward calculation.
  \item Lines 6-7 compare the scaled latency ($lat$) and throughput ($tp$) metrics against the scaled target values for latency ($lat\_goal$) and throughput ($tp\_goal$). The absolute differences between observed and target values are computed to quantify the error in latency ($error\_lat$) and throughput ($error\_tp$).
  \item Lines 8-12 determine the reward based on three distinct scenarios. Firstly, if both latency and throughput goals are achieved, a high positive reward is assigned. Secondly, if both goals are not met, a low negative reward is assigned, taking into account both latency and throughput errors. The disparity in penalties, represented by $lat\_penalty$ and $tp\_penalty$, ensures that both types of errors contribute proportionately to the overall reward. Thirdly, if only the latency goal remains unmet, a low negative reward is assigned, incorporating the latency penalty and error. Finally, if only the throughput goal is unmet, a similar low negative reward is assigned, encompassing the throughput penalty and error.
\end{enumerate}

\subsection{Training Methodology}

\subsubsection{Environment Design and Implementation}

\begin{figure}[ht]
  \centering
  \includegraphics[width=1\textwidth]{images/rl_environment_architecture.png}
  \caption{Overview of the Environment Architecture}
  \label{fig:rl_environment_architecture}
\end{figure}

% Figure \ref{fig:environment_architecture} presents an overview of the environment architecture devised for training and evaluating the RL agent. The environment comprises an interactive multi-threaded application, a batch multi-threaded application, the NVM Middleware, Intel Optane PMM, and the RL agent.

% To mimic a multi-tenant serverless scenario, we concurrently execute both applications. Serverless traces are gathered and utilized to model the workload of each application. Each application organizes requests into 1-second intervals to simulate the RL process steps. To emulate high concurrency levels, multiple threads within each application are deployed to dispatch requests to the NVM Middleware using the aforementioned API. Meanwhile, the NVM Middleware processes these requests in accordance with the workflow outlined in the preceding section. Throughout this process, the RL agent observes the environment, collects state and reward signals, and selects actions based on its policy.

The environment architecture designed for training and evaluating the RL agent is depicted in Figure \ref{fig:rl_environment_architecture}. This architecture comprises several key components, including an interactive multi-threaded application, a batch multi-threaded application, the NVM Middleware, and Intel Optane PMM.

To simulate a multi-tenant serverless scenario, both applications are executed concurrently. Workload patterns for each application are derived from collected serverless traces. To emulate high concurrency levels typical in serverless environments, multiple threads within each application are employed to dispatch requests to the NVM Middleware via the API described in Section 3.3. Meanwhile, the NVM Middleware processes these requests in accordance with the workflow outlined in Section 3.2.

\begin{figure}[ht]
  \centering
  \includegraphics[width=1\textwidth]{images/rl_sequence_flow.png}
  \caption{Agent Process flow}
  \label{fig:rl_sequence_flow}
\end{figure}

In order to model the time steps inherent in an RL process, the environment organizes the applications' requests into 1-second windows, processing one window per time step. Figure \ref{fig:rl_sequence_flow} illustrates the interactions between the RL agent and the environment at each time step. Beginning with a state observation from the preceding step, the agent communicates the intended action to the environment. Subsequently, the environment relays this action to the NVM Middleware, which then allocates resources accordingly. Upon successful execution of the action, the environment initiates processing for the next window of requests. Once all requests within the window are handled, the environment gathers metrics from the NVM Middleware and furnishes a new state observation along with a reward signal to the agent. The agent utilizes this reward to update its policy, perpetuating the iterative learning process.

\subsubsection{Function Approximation}



\begin{algorithm}[ht]
  \caption{Q-Learning Algorithm}
  \label{algo:q_learning}
  \SetAlgoLined
  \KwIn{Environment $E$, Number of Episodes $N$, Learning Rate $\alpha$, Discount Factor $\gamma$, Exploration Rate $\epsilon$}
  \KwOut{Learned Q-values}
  Initialize Q-table $Q$ with random values\;
  \For{$episode \leftarrow 1$ \KwTo $N$}{
    Initialize state $s$\;
    \For{$t \leftarrow 1$ \KwTo $T$}{
      Choose action $a$ using $\epsilon$-greedy policy based on $Q$\;
      Take action $a$, observe reward $r$ and next state $s'$\;
      Update Q-value: $Q(s,a) \leftarrow (1-\alpha) \cdot Q(s,a) + \alpha \cdot (r + \gamma \cdot \max_a Q(s',a))$\;
      Update state: $s \leftarrow s'$\;
    }
    Decrease $\epsilon$ according to exploration schedule\;
  }
\end{algorithm}


\subsubsection{Proposed Q-Learning Algorithm}

Describe environment here

\chapter[Evaluation]{Evaluation}

In this chapter, the NVM Middleware and the Q-Learning Model.

\section{Experimental Setup}

\subsection{Platform}

\begin{table}[ht]
    \centering
    \caption{Experimental Platform Specifications}
    \label{table:platform_specifications}
    \begin{tabular}{|l|l|}
      \hline
      Processor & Intel\,\textsuperscript{\tiny\textregistered} Xeon\,\textsuperscript{\tiny\textregistered} Gold 6252   \\\hline
      Sockets & 2 \\\hline
      Cores per socket & 24  \\\hline
      Threads per core & 2 \\\hline
      Numa nodes & 2 \\\hline
      CPU Frequency & 2.7 GHz (3.7 GHz Turbo frequency) \\\hline
      L1d cache & 1.5 MiB  \\\hline
      L1i cache & 1.5 MiB  \\\hline
      L2 Cache & 48 MiB  \\\hline
      L3 Cache & 71.5 MiB  \\\hline
      DRAM & 16 GB DDR4 DIMM x 6 per socket  \\\hline
      Persistent Memory & 128 GB Optane PMM x 6 per socket  \\\hline
      Operating System & Ubuntu 20.04.4 LTS (Focal Fossa)  \\\hline
      \hline
    \end{tabular}
\end{table}

The experimental platform utilized in this study is detailed in Table \ref{table:platform_specifications}. It features an Intel,\textsuperscript{\tiny\textregistered} Xeon,\textsuperscript{\tiny\textregistered} Gold 6252 processor with 2 sockets, each hosting 24 cores and 2 threads per core, totaling 2 NUMA nodes. Each socket is equipped with three memory channels, housing 16 GB DDR4 DIMMs and 128 GB Optane PMMs. In aggregate, the system comprises 192 GB of DRAM and 1.5 TB of Optane persistent memory. To mitigate the NUMA effect, one socket is designated for running the NVM Middleware threads, while the other handles the interactive and batch applications, as described in Section 3.4.3.

\subsection{Optane DC PMem Configuration}
As outlined earlier, this thesis concentrates on exploring the persistent capabilities of Optane DC PMem. Consequently, Optane DC PMem is employed in the App Direct Mode throughout our experiments. To facilitate the utilization of persistent memory, we expose it via an xfs filesystem configured in dax mode, thereby bypassing the page cache. Additionally, we enhance memory management and performance by configuring the persistent memory with huge pages (2MiB) \cite{Speeding28:online}. Lastly, we deploy a PMEMKV database with a capacity of 600GB, configured with its persistent concurrent engine.

\subsection{Workload Generators}

To execute the interactive and batch applications described in Section 3.4.3, we implement two workload generators.

\textbf{YCSB}. 

For the interactive application, we utilized traces collected from Azure Functions. The dataset, available in \cite{GitHubAz35:online}, offers a comprehensive record of Azure Function blob accesses spanning November to December 2020. Our focus was on requests occurring between November 1 and November 5, particularly those with small data access sizes (less than 1 KB), which are indicative of interactive applications.

For the batch application, we gathered traces from Wukong, a serverless parallel computing framework \cite{carver2020wukong}. Traces were obtained by executing a Single Value Decomposition for a 128x128 matrix on Wukon and capturing the I/O requests generated by the framework. These collected traces represent the behavior of a throughput-oriented serverless data-analytics application.

To further amplify the concurrency of requests directed to the NVM Middleware, we accelerated the pace of the traces by a factor of 5 compared to their original timing.

\section{Efficiency of the Workload-Aware Concurrency Control Mechanism}

Utilizing the environment delineated in Section 3.4.3, we assess the efficacy of the workload-aware concurrency control mechanism integrated into the NVM Middleware against a baseline scenario lacking any concurrency control. In the baseline scenario, the concurrency cotrol mechanism is disabled, allowing a maximum of 200 concurrent data accesses on Optane DC PMem. For this examination, we deactivate the reinforcement learning agent and system monitoring, focusing primarily on the 99th percentile latency and throughput observed by the client applications.

In this experiment, we execute YCSB and SVD Trace Replay concurrently. YCSB is configured with a zipfian request distribution, 128B data access size, and a 50/50 read-to-write ratio. Additionally, each application is run with 100 client threads.

The baseline scenario is executed without any concurrency control, alongside 42 additional tests where various combinations of interactive and batch threads within the NVM Middleware are explored. In each run, a combination of $I$ (interactive) and $B$ (batch) threads is defined and held constant throughout the run. Subsequently, the 99th percentile latency observed by the YCSB requests is recorded, and the overall throughput reported by SVD Trace Replay is captured. The results are presented in Figure \ref{fig:middleware_eval}.

\begin{figure}[ht]
  \centering
  \includegraphics[width=\textwidth,height=\textheight,keepaspectratio,angle=0]{images/middleware-latency.png}
  \includegraphics[width=\textwidth,height=\textheight,keepaspectratio,angle=0]{images/middleware-tp.png}
  \caption{Middleware Evaluation}
  \label{fig:middleware_eval}
\end{figure}

Our observations reveal that in most scenarios, both workloads derive substantial benefits from the concurrency control implemented by the NVM Middleware. Relative to the baseline, the NVM Middleware demonstrates the potential to enhance the 99th latency and throughput by up to 98\% and 86\%, respectively. Furthermore, the baseline shows that the 99th latency can vary by X\% orders of magniture, while the NVM Middleware exhibits a more controlled and predictable access latencies. Notably, the figure illustrates that the performance of applications is generally improved across most thread combinations. However, improper configuration of threads within the NVM Middleware, either too few or too many, leads to performance degradation surpassing that of the baseline. This underscores the importance for operators to meticulously select the optimal combination of threads, as an incorrect choice can yield inferior results compared to operating without any concurrency control.

A significant query stemming from these findings pertains to how an operator can determine the optimal thread combination. We observe that combining 16 interactive threads with fewer than 8 batch threads yields superior latency but fails to achieve peak throughput performance. Conversely, any combination with more than 16 batchs threads achieves peak throughput but incurs elevated access latencies. To address this dilemma, the ideal approach involves selecting the combination of interactive and batch threads that satisfies both latency and throughput SLA metrics, a topic further elaborated upon in the subsequent section.

\section{Meeting SLA performance using RL}
We now provide an evaluation of the RL-driven policies to balance the number of interactive and batch threads to meet latency and throughput SLAs. In this experiment, the goal of the NVM MIddleware is to meet pre-defined SLA objectives under changing workloads. To do this, we build four different workload phases by modifying the data access size, and read/write ratio, and client threads of the interactive and batch Trace applications described in Section 6.1.3. For each phase, we train the RL agent to learn the optimal combination of interactive and batch threads that meet pre-defined latency and throughput SLAs, while maximizing performance. Finally, we measure the agent's ability to predict and adapt to workload changes in an unknown environment where the phases are randomly alternated. 

We use the following workloads to train and test the RL agent:

Phase 1. We change the interactive traces to use a data access size of 50B, a read-to-write ratio of 80-20, and 200 concurrent client threads. We change the batch traces to use a data access size of 8k, read-to-write ratio of 50-50, and 200 concurrent client threads. 

Phase 2. We change the interactive traces to use a data access size of 50B, a read-to-write ratio of 80-20, and 150 concurrent client threads. We change the batch traces to use a data access size of 8k, read-to-write ratio of 50-50, and 320 concurrent client threads.  

Phase 3. We change the interactive traces to use a data access size of 50B, a read-to-write ratio of 80-20, and 400 concurrent client threads. We change the batch traces to use a data access size of 4k, pure read, and 200 concurrent client threads.  

Phase 4. We change the interactive traces to use a data access size of 500B, a read-to-write ratio of 10-90, and 200 concurrent client threads. We change the batch traces to use a data access size of 4k, pure read, and 200 concurrent client threads. 

\subsection{Training the RL agent}

We start the learning process by tuning the hyper-parameters of the 9 linear regression models used by the RL agent. To do this, we generate a dataset of transitions in the environment by running a non-optimal random agent on the environment. We run 150 episodes of each phase and let the random agent take random actions on the environment, logging the transitions and the rewards obtained by each transition. Since each linear regression model is supposed to approximate the value function of an action, we build 9 different sub-datasets, where each dataset contains only the transitions observed by taking a specific action. We use each sub-dataset to tune the hyperparameters of a linear regression model that represents action a, choosing from the parameters (Table \ref{table:hyperparameter_tuning}) that best fits the data. The resulting hyperparameters per model are described in Table \ref{table:per_model_parameters}.

\begin{table}[ht]
  \centering
  \caption{Hyper-parameter Tuning}
  \label{table:hyperparameter_tuning}
  % Tabular environment goes AFTER the caption!
  % \begin{adjustbox}{width=1\textwidth}
  \begin{tabular}{|l|l|l|}
    % after \\: \hline or \cline{col1-col2} \cline{col3-col4} ...
    \hline
    \thead{Type} & \thead{Parameter} & \thead{Values} \\
    \hline
    Preprocessing & Degree & 1,2,3 \\\hline
    Regression & alpha & 0.1, 0.01, 0.001, 0.0001 \\\hline
    Regression & penalty & l1, l2, elasticnet \\\hline
    Regression & loss & squared, huber, epsilon\_insensitive \\\hline
    Regression & learning rate & constant, optimal, invscaling \\\hline
    Regression & max\_iterations & 100, 1000, 10000, 100000 \\
    \hline
  \end{tabular}
% \end{adjustbox}
\end{table}

\begin{table}[ht]
  \centering
  \caption{Per-Model tuned hyper-parameters}
  \label{table:per_model_parameters}
  % Tabular environment goes AFTER the caption!
  \begin{adjustbox}{width=1\textwidth}
  \begin{tabular}{|l|l|l|}
    % after \\: \hline or \cline{col1-col2} \cline{col3-col4} ...
    \hline
    \thead{Model} & \thead{Action} & \thead{Parameters} \\
    \hline
    Model\_1 & 1 & learning\_rate: invscaling, loss: squared\_loss, alpha: 0.001, max\_iter: 10000, penalty: l2 \\\hline
    Model\_2 & 2 & learning\_rate: invscaling, loss: squared\_loss, alpha: 0.0001, max\_iter: 1000, penalty: l2 \\\hline
    Model\_3 & 3 & learning\_rate: invscaling, loss: squared\_loss, alpha: 0.001, max\_iter: 1000, penalty: elasticnet \\\hline
    Model\_4 & 4 & learning\_rate: invscaling, loss: squared\_loss, alpha: 0.0001, max\_iter: 10000, penalty: l1 \\\hline
    Model\_5 & 5 & learning\_rate: invscaling, loss: squared\_loss, alpha: 0.01, max\_iter: 100, penalty: elasticnet \\\hline
    Model\_6 & 6 & learning\_rate: invscaling, loss: squared\_loss, alpha:  0.001, max\_iter: 10000, penalty: l2 \\\hline
    Model\_7 & 7 & learning\_rate: invscaling, loss: squared\_loss, alpha: 0.01, max\_iter: 1000, penalty: elasticnet \\\hline
    Model\_8 & 8 & learning\_rate: invscaling, loss: squared\_loss, alpha: 0.001, max\_iter: 100, penalty: l1 \\\hline
    Model\_9 & 9 & learning\_rate: invscaling, loss: squared\_loss, alpha: 0.0001, max\_iter: 100, penalty: elasticnet \\
    \hline
  \end{tabular}
\end{adjustbox}
\end{table}

Using the tuned linear regressoin models, we proceed to run the Q-learning algorithm for each phase using the parameters outlined in Table \ref{table:rl_training_parameters}. We address the exploration-exploitation dilemma by starting with epsilon 1 and decaying it after each episode. This causes the agent to fully explore the state space at the beginning of the training and exploit this knowledge towards the end. We observe that different phases require different training episodes to converge to an optimal pattern. We believe this is expected given that the dataset use to pre-train the models was generated with a non-optimal policy. The random agent might have been stuck in a non-optimal loop of actions and might have not generated good training samples. Therefore, the agent requires additional training to fully capture the characteristics of these phases.

\begin{table}[ht]
  \centering
  \caption{RL Training Parameters}
  \label{table:rl_training_parameters}
  % Tabular environment goes AFTER the caption!
  % \begin{adjustbox}{width=1\textwidth}
  \begin{tabular}{|l|l|}
    % after \\: \hline or \cline{col1-col2} \cline{col3-col4} ...
    \hline
    \thead{Parameter} & \thead{Value} \\
    \hline
    episodes & Phases 1\-2: 700, Phases 3\-4: 1,000 \\\hline
    Per-episode steps & 200 \\\hline
    gamma & 0.95 \\\hline
    learning rate & 0.7 \\\hline
    epsilon & 0.9 \\\hline
    epsilon\_decay & 0.1 \\
    \hline
  \end{tabular}
% \end{adjustbox}
\end{table}

\subsection{Pattern Convergence}
For earch phase, we analyze the last three episodes to determine the combination of threads to which the agent converges. We choose the last three episodes because at that point the agent is exploiting the knowledge obtained from previous episodes.

\subsubsection*{Phase 1}
We observe that the RL agent converges to 10 interactive threads and 10 batch threads. The results are showing in figure X.

\subsubsection*{Phase 2}
We observe that the RL agent converges to 10 interactive threads and 10 batch threads. The results are showing in figure X.

\subsubsection*{Phase 3}
We observe that the RL agent converges to 10 interactive threads and 10 batch threads. The results are showing in figure X.

\subsubsection*{Phase 4}
We observe that the RL agent converges to 10 interactive threads and 10 batch threads. The results are showing in figure X.

\section{Evaluation on long-running test}
We now evaluate our trained model in a scenario where we randomly alternate the phases. We run the agent against the enviroment and run 4000 steps of the simulation, alternating a random workload every 200 steps. We compare the agent's performance against a baseline were we alternate the phases but we fixe the combination of threads in the NVM MIddleware to 15 interactive and 15 batch threads. Figure X shows the results. Table X presents a reward analysis.

\chapter[Related Work]{Related Work}

\section*{Optane DC PMem}

\section*{Serverless Storage}

Previous works have proposed the development of new storage systems to overcome the limiations of existing cloud storage offerings \cite{klimovic2018pocket,wu2019autoscaling,10.14778/3587136.3587139}. These solutions overcome the performance difference between DRAM and storage by combining multuple storage medias, keeping hot data in DRAM and cold data in persistent storage. As a Pocket \cite{klimovic2018pocket} uses application hints to decide in which tier to store the application's data. Anna \cite{} analyzes the frequency in which keys are invoked to move data between tiers. In contrast, this thesis proposes the use of the novel Optane DC PMem, which offers a unique combination of persistent storage with memory-like speeds. 

The systems mentioned above also focus on provisioning resources dynamically to meet the wrokload demands. However, they implement reactive policies, taking when certain thresholds are met. In contrast, the NVM Middleware implements a proactive approach, using Reinforcement Learning to predict workload changes and act in advance. 

\section*{Function Approximation}
% \chapter[Discussions and Future Research]{Discussions and Future Research}

% \section*{Integration with a real-world Serverless Storage Service}

% The initial scope of this thesis included the design of a simple storage server enabling communication with applications through Remote Procedure Calls (RPC). Under the hood, the server was designed to utilize the NVM Middleware to get the best performance out of Intel Optane DC PMem. However, our measurements were impacted by network overhead found in the infrastructure were the server with Optane DC PMem was located. Therefore, we leave for future research the evaluattion of the NVM MIddleware on a real-world serverless storage service. Anna \cite{wu2019anna} presents an interesting option for testing the NVM MIddleware. Anna's design allows to extend it by adding storage tier based on OPtane DC PMem, utilizing the NVM Middleware as the optimization layer for persistent memory.

% \section*{Reducing Autoscaling Overhead}

% Our current approach to scale resources consists of adding or removing up to two threads per one-second window. This approach delays the system's ability to adapt to the optiomal combination of interactive and batch threads. If the current combination of threads is off from the optimal one by a significant amount, the NVM Middleware requires several steps to take the system to the right combination. Furthremore, we believe this approach causes some of the performances spikes observed in the results.

% An interesting approach to explore is to allow the NVM MIddleware to add or remove more than two threads per time step. This could help the NVM MIddleware to adapt to workload changes faster, therefore meeting the SLAs more effcieintly. However, this approach might add more actions in the action space, increasing the complexity of the RL model.

% \section*{Choice of Workloads}

% YCSB is an excellent tool to test a variety of workloads. However, this tool might not capture the real characteristics of serverless workloads running on real-world serverless platforms. As a result, in this thesis, we focus on using I/O traces collected from real-world serverless platforms to evaluate our RL approach. However, given that we needed to create multiple stages with distinct interactive and batch workload combinations to test the autoscaling capabilities of the RL agent, we build exprimental workloads based on real-world serverless traces by modifying certain characteristics such as the block size and the read-to-write ratio.

% We believe the choice of workloads did not allow the RL agent to aciheve the desired latency SLA. One possibility is that the chosen interactive workloads did not stress the system enough, which is why the concurrency control implemented by the NVM Middleware adds extra overhead on the observed latency. A possible research endeavor is to collect more real-world serverless traces to further understand the limitations and capabilities of the NVM Middleware. However, these traces are limited and difficult to get. Furthermore, testing new workloads in our current setup is time consuming, since the RL agent needed to go over the whole learning process, starting from generating the dataset for model selection to running Q-learning episodes for each stage.

% \section*{Evaluation other Function Approximation techniques}
% %talk about model selection and deep q-learning
% Feature extraction is a complicated process. In this work, we select a few features relevant to our end goal. However, in order to cover more generic use cases, our RL model might require more features to improve its learning capabilities. The features can take many forms, such as workload characteristics, metrics within the NVM Middleware, metrics within Optane DC PMem, and general server metrics such as CPU and memory usage.

% A poor representation of features can be a limitation for linear function approximators. Another possibility is to go beyond linear function approximators by using deep neural networks to approximate the Q-fucntion. This approach, known as deep reinforcement learning, performs well in scenarios with complex feature represenations. In most cases, this approach is capable of discovering the useful features for itself \cite{russel2020ai}. It has been proven effective in complex scearios, such as playing games. such as playing games has been proven effective in complex scenarios, such as playing games \cite{mnih2013playing,silver2017mastering}. 

\chapter{Discussions and Future Research}

\textbf{Storage Service Integration:} Originally, this thesis aimed to design a simple storage server facilitating communication with applications via Remote Procedure Calls (RPC), leveraging the NVM Middleware to optimize Intel Optane DC PMem performance. However, our measurements were affected by network overhead in the infrastructure housing the server with Optane DC PMem. Thus, evaluating the NVM Middleware on a real-world serverless storage service remains a future research endeavor. Anna key-value database \cite{wu2019anna} proposes a promising option for assessing the NVM Middleware by extending it to incorporate a storage tier based on Optane DC PMem, thereby utilizing the NVM Middleware as the optimization layer for persistent memory.

\textbf{Reducing Autoscaling Overhead:} Our current approach to scaling resources involves adding or removing up to two threads per one-second window, which may delay the system's adaptation to the optimal combination of interactive and batch threads. This delay could contribute to performance spikes observed in the results. An intriguing avenue for exploration is enabling the NVM Middleware to add or remove more than two threads per time step to expedite adaptation to workload changes, potentially enhancing SLA fulfillment efficiency. However, this approach may introduce additional actions in the action space, increasing the RL model's complexity.

\textbf{Choice of Workloads:} While YCSB offers versatility in testing various workloads, its simulations may not accurately reflect real-world serverless workload characteristics. Thus, in this thesis, we rely on I/O traces from actual serverless platforms to evaluate our RL approach. However, crafting experimental workloads based on real-world serverless traces, with modified characteristics such as block size and read-to-write ratio, was necessary to test the RL agent's autoscaling capabilities under shifting workloads. Yet, our chosen workloads may have hindered the RL agent's ability to meet latency SLAs. Obtaining more real-world serverless traces could shed light on the NVM Middleware's limitations and capabilities. However, acquiring such traces is challenging and time-consuming, as the RL agent must undergo the entire learning process for each stage, from dataset generation for model selection to running Q-learning episodes.

\textbf{Exploring Other Function Approximation Techniques:} Feature extraction is pivotal but complex. While we select relevant features for our RL model, covering more generic use cases may necessitate additional features to enhance learning capabilities. These features could encompass workload characteristics, NVM Middleware metrics, Optane DC PMem internal metrics, and general server metrics like CPU and memory usage. Limitations in feature representation may impede linear function approximators' efficacy \cite{russel2020ai}. Alternatively, employing deep neural networks for function approximation \cite{Quantit3:online}, known as deep reinforcement learning, could address these challenges. Deep reinforcement learning has demonstrated success in complex scenarios by autonomously discovering useful features, as evidenced in gaming environments \cite{mnih2013playing,silver2017mastering}.

\textbf{Learning Workload Characteristics:} The current strategy employed by the NVM Middleware involves utilizing predefined hints regarding workload characteristics to allocate requests to specific queues. However, this approach may encounter scalability challenges, prompting the exploration of autonomous learning methods for workload characteristics. Several approaches can be considered, ranging from straightforward categorization based on predefined criteria to more sophisticated methods leveraging machine learning models.
% \chapter[Conclusion]{Conclusions}

% In this thesis, we investigated the use of INtel Optane DC Persistent Memory to provide a low-latency and high-throughput storage media for building a serverless storage service that can efficiently handle FaaS workloads. Our study focuses on the limited capability of INtel Optane PMem to handle accesses from multiple concurrent processes and analyzes how these limitations affect the latency and throughput exhibited by the storage media, as well as its ability to meet pre-defined service level agreements, under shifting FaaS workloads. Given the wide hetereoginity of applications running in serverless workloads, this study is focused on two type of applications: interactive applications that demand low latency and batch which demands high throuhgput. WE implement a set of applications to simulate real-world FaaS workloads to gain insights on how to tune the persistent memory device to make it ideal for using with FaaS workloads.

% Our research confirmed that running multiple threads simulating serverless functions degrades the latency and throughput exhibited by the storage media. Similar to what previous works have concluded, this implied that concurrency control must be performed over Optane DC PMem. However, the main finding is that, depending on the applications sharing the persistent memory, the performance degradation can affter some applications more than others. How much the performance degradation affects an application depends on  the performance requirements demande by each application type.

% Our findings suggest that a better approach to control concurrency is to implement independent concurrency levels for each particular type to isolate performance among the different types of applications. Such contention management mechanism must be inline with the goals of an efficient cloud storage serivce, which include transpareent autoscaling and meeting pre-defined service level agreements

% To this end, this works proposes the development of the NVM MIddleware to control the concurrency levels on intel optane pmem to achieve low latency and high throughput for serverless scenarios with interactive and batch applications sharing intel optane dc pmem. OUr experiments show the the NVM MIddleware exhibits performance benefits compared to scenarios with no concurrency or using fixed concurrency levels. First, the NVM MIddleware's workload aware concurrency control mechanism can improve the latency and thrpuhgput of interactive and batch workloads sharing Optane DC PMem by up to 80\% and 90\% respectively, compared to running these applications with no control over Optane. Furthermore, our work demonstrate that a RL model can be designed to implement an agent capable of dynamically tuning the concurrency control optimizations implementef by the NVM MIddleware to meet pre-defined service level agreements. Training the RL agent to tune the NVM Middleware concurrency control under changing workloads works better than keeping a fixed policy all the time.

% In conclusion, this work sheds the light on how to tune the concurrency level on OPtane DC PMem to achieve high performance when used as storage media for FaaS functions. By understanding the limitations of INtel Optane DC PMeme and their implications on FaaS workloads, we can further improve the functionality of the NVM MIddleware. Our methodology has some limitations that can lead to potential future research.

% Intel Optane DC Pmem Optimizations: Besides the number of concurrent threads accessing, Yang et. al. mention other factors that affect the perforance of intel optane dc pmeme. An interesting reseasrch avenue will be to study how these factors affects its performance when tested udner FaaS workloads and learn how the NVM MIddleware can be improved to target these extra limitations.

\chapter{Conclusions and Future Work}

This thesis explored the utilization of Intel Optane DC Persistent Memory to serve as a low-latency and high-throughput storage medium for constructing a serverless storage service capable of efficiently managing Function-as-a-Service (FaaS) workloads. Focusing on the inherent limitations of Intel Optane PMem in handling concurrent processes, we analyzed their impact on storage media latency, throughput, and adherence to predefined service level agreements (SLAs) under varying FaaS workloads. Considering the diverse nature of applications within serverless workloads, our investigation concentrated on two distinct application types: interactive applications requiring low latency and batch applications necessitating high throughput. We developed a set of applications to simulate real-world FaaS workloads, aiming to glean insights into optimizing persistent memory device configurations for FaaS usage scenarios.

Our research confirmed that running multiple threads to simulate serverless functions results in degraded latency and throughput performance of the storage media. Consistent with prior studies, this underscores the necessity of implementing concurrency control mechanisms over Optane DC PMem. Notably, we found that the performance degradation may disproportionately affect certain applications, depending on their specific performance requirements.

Our findings advocate for a nuanced approach to concurrency control, suggesting the implementation of independent concurrency levels tailored to each application type to mitigate performance interference interactive and batch applications. Such contention management mechanisms must align with the objectives of an efficient cloud storage service, encompassing transparent autoscaling and the fulfillment of predefined SLAs.

To address these challenges, we propose the development of the NVM Middleware to regulate concurrency levels on Intel Optane PMem, catering to low-latency and high-throughput demands in serverless scenarios involving interactive and batch applications sharing this storage medium. Our experiments demonstrate that the NVM Middleware yields performance benefits compared to scenarios lacking concurrency control or utilizing fixed concurrency levels. Specifically, our middleware's workload-aware concurrency control mechanism can enhance the latency and throughput of interactive and batch workloads sharing Optane DC PMem by up to 80\% and 90\%, respectively, compared to scenarios with no concurrency control. Furthermore, our work illustrates the efficacy of a Reinforcement Learning (RL) agent in dynamically tuning concurrency control optimizations implemented by the NVM Middleware to meet predefined SLAs. Training the RL agent to adapt the NVM Middleware's concurrency control to changing workloads outperforms static policy approaches.

In conclusion, this thesis illuminates strategies for optimizing concurrency levels on Optane DC PMem to achieve peak performance when utilized as storage media for FaaS functions. By comprehending the limitations of Intel Optane DC PMem and their ramifications on FaaS workloads, we can refine the functionality of the NVM Middleware.

\subsection*{Future Research Directions}

This section delves into the limitations inherent in our methodology and the insights gleaned from the findings presented. Subsequently, potential avenues for future research are explored in light of these limitations.

\textbf{Storage Service Integration:} Originally, this thesis aimed to design a simple storage server facilitating communication with applications via Remote Procedure Calls (RPC), leveraging the NVM Middleware to optimize Intel Optane DC PMem performance. However, our measurements were affected by network overhead in the infrastructure housing the server with Optane DC PMem. Thus, evaluating the NVM Middleware on a real-world serverless storage service remains a future research endeavor. Anna key-value database \cite{wu2019anna} proposes a promising option for assessing the NVM Middleware by extending it to incorporate a storage tier based on Optane DC PMem, thereby utilizing the NVM Middleware as the optimization layer for persistent memory.

\textbf{Reducing Autoscaling Overhead:} Our current approach to scaling resources involves adding or removing up to two threads per one-second window, which may delay the system's adaptation to the optimal combination of interactive and batch threads. This delay could contribute to performance spikes observed in the results. An intriguing avenue for exploration is enabling the NVM Middleware to add or remove more than two threads per time step to expedite adaptation to workload changes, potentially enhancing SLA fulfillment efficiency. However, this approach may introduce additional actions in the action space, increasing the RL model's complexity.

\textbf{Choice of Workloads:} While YCSB offers versatility in testing various workloads, its simulations may not accurately reflect real-world serverless workload characteristics. Thus, in this thesis, we rely on I/O traces from actual serverless platforms to evaluate our RL approach. However, crafting experimental workloads based on real-world serverless traces, with modified characteristics such as block size and read-to-write ratio, was necessary to test the RL agent's autoscaling capabilities under shifting workloads. Yet, our chosen workloads may have hindered the RL agent's ability to meet latency SLAs. Obtaining more real-world serverless traces could shed light on the NVM Middleware's limitations and capabilities. However, acquiring such traces is challenging and time-consuming, as the RL agent must undergo the entire learning process for each stage, from dataset generation for model selection to running Q-learning episodes.

\textbf{Exploring Other Function Approximation Techniques:} Feature extraction is pivotal but complex. While we select relevant features for our RL model, covering more generic use cases may necessitate additional features to enhance learning capabilities. These features could encompass workload characteristics, NVM Middleware metrics, Optane DC PMem internal metrics, and general server metrics like CPU and memory usage. Limitations in feature representation may impede linear function approximators' efficacy \cite{russel2020ai}. Alternatively, employing deep neural networks for function approximation \cite{Quantit3:online}, known as deep reinforcement learning, could address these challenges. Deep reinforcement learning has demonstrated success in complex scenarios by autonomously discovering useful features, as evidenced in gaming environments \cite{mnih2013playing,silver2017mastering}.

\textbf{Learning Workload Characteristics:} The current strategy employed by the NVM Middleware involves utilizing predefined hints regarding workload characteristics to allocate requests to specific queues. However, this approach may encounter scalability challenges, prompting the exploration of autonomous learning methods for workload characteristics. Several approaches can be considered, ranging from straightforward categorization based on predefined criteria to more sophisticated methods leveraging machine learning models.

\textbf{Intel Optane DC PMem Optimizations:} Beyond the number of concurrent threads accessing Optane DC PMem, as highlighted by Yang et al., additional factors influence the performance of Intel Optane DC PMem. An intriguing avenue for future research involves investigating how these factors impact performance under FaaS workloads and discerning ways to enhance the NVM Middleware to address these supplementary limitations.

%% include the following commands if there are any appendices
\appendix
\appendixeqnumbering
% \appchapter[Reinforcement Learning Algorithms]{Reinforcement Learning Algorithms}
\label{appendix:a}

\begin{algorithm}[ht]
    \small
    \caption{Reward Calculation Algorithm}
    \label{algo:reward_calculation}
    \SetAlgoLined
    \KwIn{System statistics: $\text{stat}$}
    \KwOut{Reward value: $\text{reward}$}
    \tcc{Initialize variables}
    $\text{max\_scale\_lat} \leftarrow 1000$, $\text{max\_scale\_tp} \leftarrow 10$, $\text{min\_scale} \leftarrow 1$, $\text{lat\_goal} \leftarrow 200$, $\text{tp\_goal} \leftarrow 250000$, $\text{lat\_penalty} \leftarrow 500.0$, $\text{tp\_penalty} \leftarrow 5000.0$\;
      
    \tcc{Scale observed and target latency and throughput}
    $\text{lat} \leftarrow ((\text{max\_scale\_lat} - \text{min\_scale}) \times (\text{stat.tailLatency} - \text{min\_value}) / (\text{max\_latency} - \text{min\_value})) + \text{min\_scale}$\;
    $\text{tp} \leftarrow ((\text{max\_scale\_tp} - \text{min\_scale}) \times (\text{stat.throughput} - \text{min\_value}) / (\text{max\_throughput} - \text{min\_value})) + \text{min\_scale}$\;
    $\text{lat\_goal} \leftarrow ((\text{max\_scale\_lat} - \text{min\_scale}) \times (\text{lat\_goal} - \text{min\_value}) / (\text{max\_latency} - \text{min\_value})) + \text{min\_scale}$\;
    $\text{tp\_goal} \leftarrow ((\text{max\_scale\_tp} - \text{min\_scale}) \times (\text{tp\_goal} - \text{min\_value}) / (\text{max\_throughput} - \text{min\_value})) + \text{min\_scale}$\;
      
    \tcc{Calculate errors}
    $\text{error\_lat} \leftarrow |\text{lat} - \text{lat\_goal}|$\;
    $\text{error\_tp} \leftarrow |\text{tp} - \text{tp\_goal}|$\;
      
    \tcc{Calculate reward}
    \eIf{$\text{lat} \leq \text{lat\_goal}$ \textbf{and} $\text{tp} \geq \text{tp\_goal}$}{
        $\text{reward} \leftarrow 10 \times (\text{error\_lat} + \text{error\_tp}$) \tcp*{High reward for meeting both latency and throughput goals}
    }{
        \eIf{$\text{lat} > \text{lat\_goal}$ \textbf{and} $\text{tp} < \text{tp\_goal}$}{
            $\text{reward} \leftarrow -1 \times (\text{lat\_penalty} \times \text{error\_lat} + \text{tp\_penalty} \times \text{error\_tp})$ \tcp*{Penalize for high latency and low throughput}
        }{
            \eIf{$\text{lat} > \text{lat\_goal}$}{
                $\text{reward} \leftarrow -1 \times \text{lat\_penalty} \times \text{error\_lat}$ \tcp*{Penalize for high latency}
            }{
                $\text{reward} \leftarrow -1 \times \text{tp\_penalty} \times \text{error\_tp}$ \tcp*{Penalize for low throughput}
            }
        }
    }
  \end{algorithm}

  \begin{algorithm}[ht]
    \small
    \caption{Q-Learning Algorithm}
    \label{algo:q_learning_mw}
    \SetAlgoLined
    \KwIn{Pre-trained Q-value models $M_a$ for all actions $a$}
    \KwOut{Learned Q-value models $M_a$ for all actions $a$}
    Initialize the training parameters $\alpha$, $\gamma$, $\epsilon$\;
    \For{$episode \leftarrow 1$ \KwTo $E$}{
      Reset the environment\;
      \Repeat{episode is done}{
        Observe the state $s_t$\;
        \tcp{Choose action $a_t$ using the $\epsilon$-greedy policy}
        Generate random number $r$ from uniform distribution in [0, 1]\;
        \If{$r < \epsilon$}{
          Select a random action $a_t$ from the action space \;
        }
        \Else{
          \For{each action $a$}{
              Predict Q-value $Q_a(s_t)$ using model $M_a$: $Q_a(s_t) \leftarrow M_a.predict(s_t)$ \;
          }
          Select action $a_t \leftarrow \argmax_a Q_a(s_t)$ \;
        }
        Take action $a_t$, observe reward $r$ and next state $s_{t+1}$\;
        \tcp{Update the Q-value model using reward and next state}
        \If{not done}{
          \For{each action $a$}{
            Predict Q-value $Q_a(s_{t+1})$ using model $M_a$: $Q_a(s_{t+1}) \leftarrow M_a.predict(s_{t+1})$ \;
          }
          Calculate target Q-value: $target \leftarrow r + \gamma \cdot \max_a Q_a(s_{t+1})$ \;
        }
        \Else{
          Set target Q-value to the reward: $target \leftarrow r$ \;
        }
        Update the model for action $a_t$ with the target Q-value: $M_{a_t}.partial\_fit(s_t, target)$ \;
        Update state: $s_t \leftarrow s_{t+1}$\;
      }
      Decrease $\epsilon$ according to exploration schedule\;
    }
  \end{algorithm}
% 
%% A sample appendix
%%
%%**********************************************************************
%% Legal Notice:
%% This code is offered as-is without any warranty either
%% expressed or implied; without even the implied warranty of
%% MERCHANTABILITY or FITNESS FOR A PARTICULAR PURPOSE!
%% User assumes all risk.
%% In no event shall any contributor to this code be liable for any damages
%% or losses, including, but not limited to, incidental, consequential, or
%% any other damages, resulting from the use or misuse of any information
%% contained here.
%%**********************************************************************
%%
%% $Id: Appendix.tex,v 1.5 2006/08/24 21:12:47 Owner Exp $
%%

% N.B.: an appendix chapter starts with "appchapter" instead of "chapter"
%
% The first argument in [ ] is the title as displayed in the table of contents
% The second argument is the title as displayed here.  Use \\ as appropriate in
%   this title to get desired line breaks
\appchapter[Concurrency Control Results]{Concurrency Control Results}
\label{appendix:b}

\begin{figure}[ht]
    \centering
    \includegraphics[width=\textwidth,height=\textheight,keepaspectratio,angle=0]{images/middleware-latency.png}
    \includegraphics[width=\textwidth,height=\textheight,keepaspectratio,angle=0]{images/middleware-tp.png}
    \caption{Middleware Evaluation}
    \label{fig:middleware_eval}
  \end{figure}

% \appchapter[Phases]{Phases}
\label{appendix:c}

We use the following workloads to train and test the RL agent:

Phase 1. We change the interactive traces to use a data access size of 50B, a read-to-write ratio of 80-20, and 200 concurrent client threads. We change the batch traces to use a data access size of 8k, read-to-write ratio of 50-50, and 200 concurrent client threads. 

Phase 2. We change the interactive traces to use a data access size of 50B, a read-to-write ratio of 80-20, and 150 concurrent client threads. We change the batch traces to use a data access size of 8k, read-to-write ratio of 50-50, and 320 concurrent client threads.  

Phase 3. We change the interactive traces to use a data access size of 50B, a read-to-write ratio of 80-20, and 400 concurrent client threads. We change the batch traces to use a data access size of 4k, pure read, and 200 concurrent client threads.  

Phase 4. We change the interactive traces to use a data access size of 500B, a read-to-write ratio of 10-90, and 200 concurrent client threads. We change the batch traces to use a data access size of 4k, pure read, and 200 concurrent client threads. 


% \appchapter[Reinforcement Learning Agent Training]{Reinforcement Learning Agent Training}
\label{appendix:d}

% Phase 1
% We observe that the RL agent converges to 10 interactive threads and 10 batch threads. The results are showing in figure X.

% Phase 2
% We observe that the RL agent converges to 10 interactive threads and 10 batch threads. The results are showing in figure X.

% Phase 3
% We observe that the RL agent converges to 10 interactive threads and 10 batch threads. The results are showing in figure X.

% Phase 4
% We observe that the RL agent converges to 10 interactive threads and 10 batch threads. The results are showing in figure X.

\begin{table}[ht]
    \centering
    % \caption[RL Regression Models: Hyperparameter Tuning]{List of parameters used for tuning hyperparameters of the polynomial regression models used for function approximation. $Degree$ represents the degree of the polynomial function. $Loss$ represents the loss funciton used by stochastic gradient descent. regularizer represents the penalty to applied to the loss function to shrink the model parameters towards the zero vector. This technique is widely used in regression models to avoid overfitting the data. $Alpha$ represent the strenght to whic the regularization penalty is applied. $learning\_rate$ represents the learning rate schedul of the model. $Max_iter$ represents the maximum number of passes over the training dat}
    \caption[Hyperparameter Tuning Options for Polynomial Regression Models]{Overview of hyperparameters used in the tuning process for polynomial regression models employed in function approximation. The $degree$ parameter denotes the degree of the polynomial function utilized. The $loss$ parameter specifies the loss function employed during stochastic gradient descent. The $penalty$ parameter represents the regularization technique applied to mitigate overfitting. The $alpha$ parameter indicates the strength of the regularization penalty. The $learning\_rate$ parameter determines the scheduling of the model's learning rate. Finally, the $max\_iterations$ parameter sets the maximum number of passes over the training data.}
    \label{table:hyperparameter_tuning}
    % Tabular environment goes AFTER the caption!
    % \begin{adjustbox}{width=1\textwidth}
    \begin{tabular}{|c|c|}
      % after \\: \hline or \cline{col1-col2} \cline{col3-col4} ...
      \hline
      \thead{Parameter} & \thead{Values} \\
      \hline
      degree & 1,2,3 \\\hline
      loss & squared, huber, epsilon\_insensitive \\\hline
      penalty & l1, l2, elasticnet \\\hline
      alpha & 0.1, 0.01, 0.001, 0.0001 \\\hline
      learning rate & constant, optimal, invscaling \\\hline
      max\_iterations & 100, 1000, 10000, 100000 \\
      \hline
    \end{tabular}
    % \caption{Model Selection and Hyper-parameter Tuning parameters categorized by type.}
  % \end{adjustbox}
  \end{table}
  
  \begin{table}[ht]
    \centering
    \caption[Resulting Hyperparameters for Polynomial Regression Models]{Hyperparameters of the polynomial regression models after hyperparameter tuning.}
    \label{table:per_model_parameters}
    % Tabular environment goes AFTER the caption!
    \begin{adjustbox}{width=1\textwidth}
    \begin{tabular}{|l|l|}
      % after \\: \hline or \cline{col1-col2} \cline{col3-col4} ...
      \hline
      \thead{Model} & \thead{Parameters} \\
      \hline
      Model\_1 & \makecell[cl] {degree: 2, learning\_rate: 'invscaling', loss: 'squared\_loss', alpha: 0.1, max\_iter: 1000, \\ penalty: elasticnet} \\\hline
        Model\_2 & \makecell[cl] {degree: 2, learning\_rate: 'invscaling', loss: 'squared\_loss', alpha: 0.01, max\_iter: 10000, \\ penalty: l1} \\\hline
        Model\_3 & \makecell[cl] {degree: 2, learning\_rate: 'invscaling', loss: 'squared\_loss', alpha: 0.0001, max\_iter: 100, \\ penalty: elasticnet} \\\hline
        Model\_4 & \makecell[cl] {degree: 2, learning\_rate: 'invscaling', loss: 'squared\_loss', alpha: 0.001, max\_iter: 10000, \\ penalty: l1} \\\hline
          Model\_5 & \makecell[cl] {degree: 2, learning\_rate: 'invscaling', loss: 'squared\_loss', alpha: 0.01, max\_iter: 10000, \\ penalty: elasticnet} \\\hline
      Model\_6 & \makecell[cl] {degree: 2, learning\_rate: 'invscaling', loss: 'squared\_loss', alpha:  0.01, max\_iter: 1000, \\ penalty: elasticnet} \\\hline
      Model\_7 & \makecell[cl] {degree: 2, learning\_rate: 'invscaling', loss: 'squared\_loss', alpha: 0.01, max\_iter: 1000, \\ penalty: elasticnet} \\\hline
      Model\_8 & \makecell[cl] {degree: 2, learning\_rate: 'invscaling', loss: 'squared\_loss', alpha: 0.1, max\_iter: 10000, \\ penalty: l1} \\\hline
      Model\_9 & \makecell[cl] { degree: 2, learning\_rate: 'invscaling', loss: 'squared\_loss', alpha: 0.01, max\_iter: 100, \\ penalty: elasticnet} \\
      \hline
    \end{tabular}
  \end{adjustbox}
  \end{table}
  
  \begin{table}[ht]
    \centering
    \caption[Q-Learning Parameters]{Parameters utilized for Q-Learning by the RL agent. The target 99th percentile latency and throughput represent predefined SLAs guiding the reward calculation. The aim is to maintain the observed 99th percentile latency below 250 microseconds while ensuring throughput remains above 250,000 operations per second. The exploration rate ($epsilon$) diminishes gradually between episodes, following a decay rate ($epsilon\_decay$), to strike a balance between exploration and exploitation of knowledge. Furthermores, additional episodes beyond the initially specified count are introduced to enable the agent to reach the optimal policy for phases 1, 3, and 4.}
    \label{table:rl_training_parameters}
    % Tabular environment goes AFTER the caption!
    % \begin{adjustbox}{width=1\textwidth}
    \begin{tabular}{|c|c|}
      % after \\: \hline or \cline{col1-col2} \cline{col3-col4} ...
      \hline
      \thead{Parameter} & \thead{Value} \\
      \hline
      episodes & Phases $1,3,4$: 1,000, Phase $2$: 700 \\\hline
      steps per episode & 200 \\\hline
      gamma & 0.95 \\\hline
      learning rate & 0.7 \\\hline
      epsilon & 0.9 \\\hline
      epsilon\_decay & 0.1 \\\hline
      Target 99th Latency & $\leq$ 250 microseconds \\\hline
      Target throughput & $\geq$ 250,000 operations/second \\
      \hline
    \end{tabular}
  % \end{adjustbox}
  \end{table}

  \begin{table}[ht]
    \centering
    % \caption{Preliminary Rewards Phase 1}
    \caption[Preliminary Measurements for Phase 1]{Experimental reward analysis conducted on Phase 1 using the NVM Middleware with various fixed combinations of interactive ($I$) and batch ($B$) threads. The table displays statistics on rewards obtained under different configurations, denoted as I5/B5, I10/B10, I15/B15, I15/B5, and I5/B15. The values represent the distribution of rewards, ranging from the minimum to the maximum observed, along with quartiles and median scores. Based on these preliminary findings, the configuration with 10 interactive threads and 10 batch threads appears to yield the most favorable results for Phase 1.}
    \label{table:rewards_phase_1}
    % Tabular environment goes AFTER the caption!
    % \begin{adjustbox}{width=1\textwidth}
    \begin{tabular}{|c|c|c|c|c|c|}
      % after \\: \hline or \cline{col1-col2} \cline{col3-col4} ...
      \hline
      \thead{} & \thead{I5/B5} & \thead{I10/B10} & \thead{I15/B15} & \thead{I15/B5} & \thead{I5/B15}\\
      \hline
      Min & -5716 & \cellcolor{green}-8312 & -41660 & -5682 & -9436\\\hline
      Q1 & -1383 & \cellcolor{green}2.7 & -4797 & -2971 & -1\\\hline
      Median & -171 & \cellcolor{green}3.92 & -2 & -1850 & 0\\\hline
      Q3 & 0 & \cellcolor{green}4.3 & 3 & -1035 & 1\\\hline
      Max & 1 & \cellcolor{green}5 & 5 & 3 & 2\\
      \hline
    \end{tabular}
  % \end{adjustbox}
  \end{table}

  \begin{table}[ht]
    \centering
    % \caption{Preliminary Rewards Phase 2}
    \caption[Preliminary Measurements for Phase 2]{Experimental reward analysis conducted on Phase 2 using the NVM Middleware with various fixed combinations of interactive ($I$) and batch ($B$) threads. The table displays statistics on rewards obtained under different configurations, denoted as I5/B5, I10/B10, I15/B15, I15/B5, and I5/B15. The values represent the distribution of rewards, ranging from the minimum to the maximum observed, along with quartiles and median scores. Based on these preliminary findings, the configuration with 10 interactive threads and 10 batch threads appears to yield the most favorable results for Phase 2.}
    \label{table:rewards_phase_2}
    % Tabular environment goes AFTER the caption!
    % \begin{adjustbox}{width=1\textwidth}
    \begin{tabular}{|c|c|c|c|c|c|}
      % after \\: \hline or \cline{col1-col2} \cline{col3-col4} ...
      \hline
      \thead{} & \thead{I5/B5} & \thead{I10/B10} & \thead{I15/B15} & \thead{I15/B5} & \thead{I5/B15}\\
      \hline
      Min & -4320 & \cellcolor{green}-7763 & -20455 & -5328 & -10060\\\hline
      Q1 & -463 & \cellcolor{green}-1.9 & -1169 & -368 & 1\\\hline
      Median & 1.3 & \cellcolor{green}5.2 & 4.2 & 3.1 & 3\\\hline
      Q3 & 1.8 & \cellcolor{green}5.8 & 4.9 & 3.3 & 4\\\hline
      Max & 2.8 & \cellcolor{green}6 & 6 & 3.8 & 6\\
      \hline
    \end{tabular}
  % \end{adjustbox}
  \end{table}

  \begin{table}[ht]
    \centering
    % \caption{Preliminary Rewards Phase 3}
    \caption[Preliminary Measurements for Phase 3]{Experimental reward analysis conducted on Phase 3 using the NVM Middleware with various fixed combinations of interactive ($I$) and batch ($B$) threads. The table displays statistics on rewards obtained under different configurations, denoted as I5/B5, I7/B3, I7/B7, I10/B10, I15/B15, I15/B5, and I5/B15. The values represent the distribution of rewards, ranging from the minimum to the maximum observed, along with quartiles and median scores. Based on these preliminary findings, the configuration with 7 interactive threads and 3 batch threads appears to yield the most favorable results for Phase 3.}
    \label{table:rewards_phase_3}
    % Tabular environment goes AFTER the caption!
    % \begin{adjustbox}{width=1\textwidth}
    \begin{tabular}{|c|c|c|c|c|c|c|c|}
      % after \\: \hline or \cline{col1-col2} \cline{col3-col4} ...
      \hline
      \thead{} & \thead{I5/B5} & \thead{I7/B3} & \thead{I7/B7} & \thead{I10/B10} & \thead{I15/B15} & \thead{I15/B5} & \thead{I5/B15}\\
      \hline
      Min & -12490 & \cellcolor{green}-6582 & -174852 & -141354 & -149647 & -96900 & -13211\\\hline
      Q1 & -1230 & \cellcolor{green}-1727 & -2567 & -18768 & -42951 & -58414 & -8583\\\hline
      Median & -272 & \cellcolor{green}-672 & -4 & -566 & -14008 & -2954 & -4940\\\hline
      Q3 & -5 & \cellcolor{green}-3 & -2 & -1 & -4607 & 0 & -2709\\\hline
      Max & -3 & \cellcolor{green}-1 & 0 & 3 & -1 & 4 & -943\\
      \hline
    \end{tabular}
  % \end{adjustbox}
  \end{table}

  \begin{table}[ht]
    \centering
    % \caption{Preliminary Rewards Phase 4}
    \caption[Preliminary Measurements for Phase 4]{Experimental reward analysis conducted on Phase 4 using the NVM Middleware with various fixed combinations of interactive ($I$) and batch ($B$) threads. The table displays statistics on rewards obtained under different configurations, denoted as I5/B5, I10/B10, I15/B15, and I15/B5. The values represent the distribution of rewards, ranging from the minimum to the maximum observed, along with quartiles and median scores. Based on these preliminary findings, the configuration with 15 interactive threads and 5 batch threads appears to yield the most favorable results for Phase 4.}
    \label{table:rewards_phase_4}
    % Tabular environment goes AFTER the caption!
    % \begin{adjustbox}{width=1\textwidth}
    \begin{tabular}{|c|c|c|c|c|}
      % after \\: \hline or \cline{col1-col2} \cline{col3-col4} ...
      \hline
      \thead{} & \thead{I5/B5} & \thead{I10/B10} & \thead{I15/B15} & \thead{I15/B5}\\
      \hline
      Min & -11341 & -14699 & -21259 & \cellcolor{green}-3912\\\hline
      Q1 & -153 & -2.5 & -6908 & \cellcolor{green}-1\\\hline
      Median & -2 & 2.3 & -3117 & \cellcolor{green}3\\\hline
      Q3 & 0 & 3.7 & -4 & \cellcolor{green}4\\\hline
      Max & 3 & 5 & 3 & \cellcolor{green}5\\
      \hline
    \end{tabular}
  % \end{adjustbox}
  \end{table}

  \begin{figure}[ht]
    \centering
    \includegraphics[width=\textwidth,height=\textheight,keepaspectratio,angle=0]{images/rl_training_phase1.png}
    % \caption[Phase 1: Agent's Learned Pattern]{Analysis of the three last episodes of the Q-Learning process performed by the agent when running Phase 1 for 1,000 episodes. On these episodes, the exploration rate is low, indicating that the agent is exploiting its knowledge acquired from previous episodes. The first row demonstrated that the agent configures the NVM Middleware with 10 interactive and 10 batch threads when it starts receiving requests sent from applications configured in Phase 1. This combination of threads matches the preliminary resutls obtained for this phase. Additionally, the second row demonstrates how by configuring the NVM Middleware with the optiomal combination of threads, the agent, most of the steps, keeps the latency low (less than 250 microseconds) and the throughput below 250,000 operations/second.}
    \caption[Learned Pattern of Agent during Phase 1]{Visualization depicting the learned pattern of the agent during Phase 1 of the training process. The analysis focuses on the behavior observed in the final three episodes of the Q-Learning process, which spanned 1,000 episodes. During these episodes, the exploration rate is so low that the agent predominantly exploits its accumulated knowledge. The first row illustrates the agent's configuration of the NVM Middleware with approximately 10 interactive and 10 batch threads, aligning with preliminary results for Phase 1. In the middle row, the throughput and 99th percentile latency reported by the NVM Middleware at each time step are depicted. By employing the optimal combination of threads, the agent consistently maintains low latency (less than 250 microseconds) and a throughput exceeding 250,000 operations/second across most steps. Finally, the bottom row illustrates the operations per second sent to the NVM Middleware by Phase 1.}
    \label{fig:learned_phase_1}
  \end{figure}

  \begin{figure}[ht]
    \centering
    \includegraphics[width=\textwidth,height=\textheight,keepaspectratio,angle=0]{images/rl_training_phase2.png}
    % \caption{Learned Pattern Phase 2}
    \caption[Learned Pattern of Agent during Phase 2]{Visualization depicting the learned pattern of the agent during Phase 2 of the training process. The analysis focuses on the behavior observed in the final three episodes of the Q-Learning process, which spanned 700 episodes. During these episodes, the exploration rate is so low that the agent predominantly exploits its accumulated knowledge. The first row illustrates the agent's configuration of the NVM Middleware with high number of interactive and batch threads (between 10-15), aligning with preliminary results for Phase 2. In the middle row, the throughput and 99th percentile latency reported by the NVM Middleware at each time step are depicted. By employing the optimal combination of threads, the agent consistently maintains low latency (less than 250 microseconds) and a throughput exceeding 250,000 operations/second across most steps. Finally, the bottom row illustrates the operations per second sent to the NVM Middleware by Phase 2.}
    \label{fig:learned_phase_2}
  \end{figure}

  \begin{figure}[ht]
    \centering
    \includegraphics[width=\textwidth,height=\textheight,keepaspectratio,angle=0]{images/rl_training_phase3.png}
    % \caption{Learned Pattern Phase 3}
    \caption[Learned Pattern of Agent during Phase 3]{Visualization depicting the learned pattern of the agent during Phase 1 of the training process. The analysis focuses on the behavior observed in the final three episodes of the Q-Learning process, which spanned 1,000 episodes. During these episodes, the exploration rate is so low that the agent predominantly exploits its accumulated knowledge. The first row illustrates the agent's configuration of the NVM Middleware with approximately 8 interactive and low number of batch threads (approximately less than 5), aligning with preliminary results for Phase 3. In the middle row, the throughput and 99th percentile latency reported by the NVM Middleware at each time step are depicted. By employing the optimal combination of threads, the agent consistently maintains low latency (less than 250 microseconds) and a throughput exceeding 250,000 operations/second across most steps. Finally, the bottom row illustrates the operations per second sent to the NVM Middleware by Phase 3.}
    \label{fig:learned_phase_3}
  \end{figure}

  \begin{figure}[ht]
    \centering
    \includegraphics[width=\textwidth,height=\textheight,keepaspectratio,angle=0]{images/rl_training_phase4.png}
    % \caption{Learned Pattern Phase 4}
    \caption[Learned Pattern of Agent during Phase 3]{Visualization depicting the learned pattern of the agent during Phase 1 of the training process. The analysis focuses on the behavior observed in the final three episodes of the Q-Learning process, which spanned 1,000 episodes. During these episodes, the exploration rate is so low that the agent predominantly exploits its accumulated knowledge. The first row illustrates the agent's configuration of the NVM Middleware with high number of interactive threads (approximately 15) and low number of batch threads (approximately 5), aligning with preliminary results for Phase 4. In the middle row, the throughput and 99th percentile latency reported by the NVM Middleware at each time step are depicted. By employing the optimal combination of threads, the agent consistently maintains low latency (less than 250 microseconds) and a throughput exceeding 250,000 operations/second across most steps. Finally, the bottom row illustrates the operations per second sent to the NVM Middleware by Phase 4.}
    \label{fig:learned_phase_4}
  \end{figure}
% \appchapter[RL Agent Evaluation]{RL Agent Evaluation}
\label{appendix:e}

\begin{figure}[ht]
  \centering
  \includegraphics[width=\textwidth,height=\textheight,keepaspectratio]{images/long_run_sim.png}
  \caption[RL Agent Adaptation to Shifting Workloads]{Illustration demonstrating the dynamic adjustment of thread counts by the RL agent during a test comprising 4,000 steps with randomly alternating phases. The top row showcases the RL agent's utilization of learned knowledge to optimize the number of interactive and batch threads for each phase. In the middle row, the throughput and 99th percentile latency reported by the NVM Middleware at each time step are depicted. By dynamically modifying the NVM Middleware threads, the throughput aligns with the predefined throughput SLA. However, unexpected spikes in the 99th percentile latency are observed, contrary to the training phase. The bottom row illustrates the operations per second sent by each phase, highlighting the shift in patterns every 200 steps.}
  \label{fig:long_run_eval}
\end{figure}

\begin{table}[ht]
    \centering
    \caption[RL Agent Reward Analysis in Long-run Test]{Analysis of the rewards achieved by the RL agent under shifting workloads compared to two baseline scenarios: one without concurrency control and the other keeping the NVM Middleware threads fixed. The values represent the distribution of reward signals reported by the environment at each time step, ranging from the minimum to the maximum observed, along with quartiles and median scores. Overall, the NVM Middleware with the RL agent achieves the highest reward.}
    \label{table:eval_results_reward}
    % Tabular environment goes AFTER the caption!
    % \begin{adjustbox}{width=1\textwidth}
    \begin{tabular}{|c|c|c|c|}
      % after \\: \hline or \cline{col1-col2} \cline{col3-col4} ...
      \hline
      \thead{} & \thead{No NVM Middleware} & \thead{NVM Middleware Fixed} & \thead{NVM Middleware + RL} \\
      \hline
      Min & -94755.386 & -65157.089 & -87907.434 \\\hline
      Q1 & -10790.1 & -8605.4625 & -3647.9454 \\\hline
      Median & -6983.235 & -3208.995 & -4.005294 \\\hline
      Q3 & -4042.9125 & -2.353015 & 0.31523 \\\hline
      Max & 3.174046 & 4.89667 & 4.582787 \\
      \hline
    \end{tabular}
  % \end{adjustbox}
\end{table}

\begin{table}[ht]
    \centering
    \caption[RL Agent Throughput Analysis in Long-run Test]{Analysis of the throughput achieved by the RL agent under shifting workloads compared to two baseline scenarios: one without concurrency control and the other keeping the NVM Middleware threads fixed. The values represent the distribution of throughput measurements reported by the NVM Middleware at each time step, ranging from the minimum to the maximum observed, along with quartiles and median scores. Overall, the NVM Middleware with the RL agent achieves the highest throughput.}
    \label{table:eval_results_tp}
    % Tabular environment goes AFTER the caption!
    % \begin{adjustbox}{width=1\textwidth}
    \begin{tabular}{|c|c|c|c|}
      % after \\: \hline or \cline{col1-col2} \cline{col3-col4} ...
      \hline
      \thead{} & \thead{No NVM Middleware} & \thead{NVM Middleware Fixed} & \thead{NVM Middleware + RL} \\
      \hline
      Min & 72,218.9 & 104,719.9 & 159,168.8 \\\hline
      Q1 & 112,440 & 148,792.5 & 218,816.25 \\\hline
      Median & 135,508.5 & 186,877 & 252,078.5 \\\hline
      Q3 & 166,340.25 & 231,597.75 & 278,205 \\\hline
      Max & 242,104.4 & 357,445.6 & 352,800.27 \\
      \hline
    \end{tabular}
  % \end{adjustbox}
\end{table}

\begin{table}[ht]
    \centering
    \caption[RL Agent Latency Analysis in Long-run Test]{Analysis of the throughput achieved by the RL agent under shifting workloads compared to two baseline scenarios: one without concurrency control and the other keeping the NVM Middleware threads fixed. The values represent the distribution of latency measurements reported by the NVM Middleware at each time step, ranging from the minimum to the maximum observed, along with quartiles and median scores. Surprisingly, in this particular test, both configurations involving the NVM Middleware exhibited higher latency compared to the baseline without concurrency control. This unexpected outcome suggests that the interactive workloads may not have adequately stressed the system, resulting in increased latency when utilizing the NVM Middleware. Further investigation is warranted to ascertain the underlying factors contributing to this behavior.}
    \label{table:eval_results_latency}
    % Tabular environment goes AFTER the caption!
    % \begin{adjustbox}{width=1\textwidth}
    \begin{tabular}{|c|c|c|c|}
      % after \\: \hline or \cline{col1-col2} \cline{col3-col4} ...
      \hline
      \thead{} & \thead{No NVM Middleware} & \thead{NVM Middleware Fixed} & \thead{NVM Middleware + RL} \\
      \hline
      Min & 19 & 89 & 130.95 \\\hline
      Q1 & 81 & 239 & 396.75 \\\hline
      Median & 19 & 674.5 & 754 \\\hline
      Q3 & 271.5 & 1888.75 & 1427 \\\hline
      Max & 82824 & 34090.93 & 34712.04 \\
      \hline
    \end{tabular}
  % \end{adjustbox}
\end{table}

%%
%%  bibliography
%%

%% Use the bibliographicstyle command to use a .bst file for your citation style. If you use biblatex, comment this out and use it as directed in the package documentation.
\bibliographystyle{unsrt}

% The .bib file is gmuETD.bib
\bibliography{gmuETD}

%%
%% biography
%%
\biography

\noindent Rafael Madrid is a graduate student at George Mason University, pursuing a Master of Science degree in Computer Science. He holds a Bachelor of Science degree in Computer Systems Engineering from the Central American Technological University. His main research interests include machine learning, cloud computing, storage systems, and distributed systems. 
\end{document}

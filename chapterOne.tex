
%% This file represents a sample first chapter of the main body of the dissertation
%%
%%**********************************************************************
%% Legal Notice:
%% This code is offered as-is without any warranty either
%% expressed or implied; without even the implied warranty of
%% MERCHANTABILITY or FITNESS FOR A PARTICULAR PURPOSE!
%% User assumes all risk.
%% In no event shall any contributor to this code be liable for any damages
%% or losses, including, but not limited to, incidental, consequential, or
%% any other damages, resulting from the use or misuse of any information
%% contained here.
%%**********************************************************************
%%
%% $Id: chapterOne.tex,v 1.6 2006/08/24 21:13:45 Owner Exp $
%%

% A first, optional argument in [ ] is the title as displayed in the table of contents
% The second argument is the title as displayed here.  Use \\ as appropriate in
%   this title to get desired line breaks
\chapter[Introduction]{Introduction}

\section{Motivation}

Serverless computing is an increasingly popular cloud execution model that liberates application developers from the burden of traditional infrastructure management. With serverless platforms (e.g., AWS Lambda, Google Cloud Functions, Azure Functions), developers solely focus on writing their code as event-driven functions that will execute on-demand in response to events or triggers. Cloud providers are responsible for dynamically allocating and scaling resources to meet demands as the event triggers occur. With a pay-as-you-go pricing model, users only pay for the resource consumed during their function invocations, making serverless computing a cost-effective solution.

Cloud providers designed serverless functions to be stateless, meaning that they do not retain state between function invocations. This intentional statelessness is a fundamental aspect for achieving high elasticity. By eliminating the need to store state within the function invocation, serverless platforms promote scalability and ease of deployment. Cloud providers can execute functions in parallel, allowing for efficient resource utilization. Any data needed between function invocations must be stored in remote storage.

Although the stateless nature of serverless computing is key to achieve high elasticity, it limits the type of applications that can run efficiently on serverless platforms. Previous studies \cite{jonas2019cloud,pu2019shuffling} have found that data-intensive applications running in serverless platforms (i.e., data analytics, ML workflows, databases) are limited by the capacity and performance gaps that exist among the existing storage services. Object storage services, such as AWS S3, provide cheap long-term storage, but exhibits high access latencies. On the other hand, in-memory clusters, such as AWS ElastiCache, exhibit low access latencies and high throughput, but they are expensive and are not transparently provisioned. In between, key-value databases, such as AWS DynamoDB, provide high throughput, but are expensive and can take a long time to scale.

Given the limitations of existing storage solutions, previous works motivate the development of a serverless storage service capable of handling the wide variety of workloads running on serverless platforms. These studies mention three requirements that such service must meet. First, it should provide low latency and high throughput for a wide range of object size and data access patterns. Second, it should be transparently provisioned and should scale to meet workload demands. Third, it must ensure isolation and predictable performance across applications and tenants.

To meet the first requirement, cloud providers must first close the capacity and performance gap between main memory and persistent storage media. As mentioned above, existing storage service have fixed tradeoffs that reflect the traditional memory hierarchy built from RAM, flash memory, and magnetic disk drives. Leveraging Non-volatile memory is a promising approach to bridge the gap between the memory and storage tiers. Non-volatile memory combines the persistence and capacity of traditional storage with the low latency and byte addressability of main memory. This technology experienced a breakthrough with the release of Intel Optane DC Persistent Memory.

Non-volatile memory technology experienced a breakthrough with the release of Intel Optane DC Persistent Memory Module (PMM). Optane PMM is an emerging technology where non-volatile media is placed in a Dual In-Line Memory Module (DIMM) and installed on the memory bus, alongside traditional DRAM (Dynamic Random Access Memory). Similar to DRAM, this technology presents a byte-addressable interface and achieves speeds comparable to DRAM (2x-3x lower). The main difference between the two is that Optane PMM has higher capacities and can retain data when the system is shutdown or loses power. This allows Optane PMM to be used as a form of persistent storage with memory-like speeds.

The unique combination of persistence and low access latency makes Optane PMM an ideal candidate to speed up data-intensive workloads running in serverless platforms. Thus, thesis presents an analysis on how to make efficient use of Optane PMM to build a serverless storage service.

\section{Research Questions}

With the release of Intel Optane DIMM, researchers have started to understand its characteristics, capabilities, and limitations \cite{izraelevitz2019basic, yang2020empirical, wu2020ribbon}. The initial expectation was that Intel Optane DC PMM would behave similar to DRAM, but with a lower performance (higher latency and lower bandwidth). However, recent studies suggest that it should not be treated as a “slower, persistent DRAM”. Compared to DRAM, Optane DC PMM exhibits complicated behaviors and its performance changes based on multiple factors, such as the access size, access type (read vs. write), and degree of concurrency.

Intel Optane DC PMM differs from DRAM in two ways. First, there is a mismatch between the CPU cacheline access granularity (64-byte) and the 3D-XPoint media access granularity (256-byte) in Intel Optane DC PMM. This difference can lead to write or read amplification if the data access is smaller than 256 bytes. Second, to balance the gap in access granularity, the Intel Optane DC PMM implements a small (16KB) write-combining buffer to merge small writes and reduce write amplification. However, the buffer’s limited capacity (16 KB) can cause contention within the device, limiting its ability to handle access from multiple threads simultaneously.

The complex behavior of Intel Optane DC PMM introduces interesting challenges for building a serverless storage service using this technology. Previous works have found that serverless functions vary considerable in multiple ways, including the way they access and process data, and their quality-of-service (QoS) demands. Furthermore, these workloads can spike by orders of magnitude and change dramatically over time. Knowing how these large-scale variations affect the system’s performance and QoS for applications can assist in building an efficient serverless storage service.

Consequently, this thesis addresses the following research questions:

\begin{itemize}
    \item How does Optane PMM affect the system’s performance when used as persistent storage for serverless functions?
    \item How does Optane PMM performance under serverless workloads affect the (QoS) for applications?
    \item How can we overcome the limitations of Optane PMM to make efficient use of the device in a serverless scenario?
    \item How do we keep the system optimized and compliant with QoS requirements over time as workload shifts occur?
\end{itemize}

\section{Contributions}

The experiments described in Section 3 provide various helpful insights on the Optane PMM behavior when used as persistent storage for serverless workloads. First, we discover that sharing the Optane PMM among hundreds of serverless functions lead to performance loss (higher latency and lower bandwidth) in the device. This fact was expected given the contention issues experienced by Optane PMM with higher thread counts. Second, we discover that, depending on the workloads, the performance degradation in Optane PMM affects one performance metric more than the other (latency vs. bandwidth). This suggests that QoS of some applications might be affected more than others. Therefore, we conclude the success of Optane PMM should be measured by its capability of meeting the QoS requirements of the current workload.

To help alleviate the limitations of Intel Optane PMM, we introduce a control layer that runs on top of Optane and guides the efficient use of the device under dynamic workloads.  Our control layer, called NVM Middleware, is designed to limit the access to persistent memory to reduce its contention. While doing so, the NVM Middleware keeps track of the type of applications running in the system and applies different optimization policies for each one to ensure that their QoS requirements are met. Using machine learning, the NVM Middleware learns how to scale resources to meet the current demand and dynamically adapts them to changing workloads. We propose using online reinforcement learning algorithms, given that data access patterns in serverless workloads can change over time.

\begin{itemize}
    \item We present an experimental study that describes the capabilities and limitations of Intel Optane PMM when used as persistent storage for serverless workloads. To our knowledge, Optane PMM has not been tested yet in this scenario.
    \item We present the NVM Middleware, a control layer promotes the efficient use of Optane PMM, while ensuring that QoS requirements for different type of applications are met.
    \item We propose a Reinforcement Learning model and framework that allows the NVM Middleware to learn from historical data and adapt resources to changing workloads.
    \item Finally, we present empirical results that demonstrate the benefits of our solution.
\end{itemize}
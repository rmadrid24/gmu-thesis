\chapter[Evaluation]{Evaluation}

In this chapter, the NVM Middleware and the Q-Learning Model.

\section{Experimental Setup}

\subsection{Platform}

\begin{table}[ht]
    \centering
    \caption{Experimental Platform Specifications}
    \label{table:platform_specifications}
    \begin{tabular}{|l|l|}
      \hline
      Processor & Intel\,\textsuperscript{\tiny\textregistered} Xeon\,\textsuperscript{\tiny\textregistered} Gold 6252   \\\hline
      Sockets & 2 \\\hline
      Cores per socket & 24  \\\hline
      Threads per core & 2 \\\hline
      Numa nodes & 2 \\\hline
      CPU Frequency & 2.7 GHz (3.7 GHz Turbo frequency) \\\hline
      L1d cache & 1.5 MiB  \\\hline
      L1i cache & 1.5 MiB  \\\hline
      L2 Cache & 48 MiB  \\\hline
      L3 Cache & 71.5 MiB  \\\hline
      DRAM & 16 GB DDR4 DIMM x 6 per socket  \\\hline
      Persistent Memory & 128 GB Optane PMM x 6 per socket  \\\hline
      Operating System & Ubuntu 20.04.4 LTS (Focal Fossa)  \\\hline
      \hline
    \end{tabular}
\end{table}

The experimental platform utilized in this study is detailed in Table \ref{table:platform_specifications}. It features an Intel,\textsuperscript{\tiny\textregistered} Xeon,\textsuperscript{\tiny\textregistered} Gold 6252 processor with 2 sockets, each hosting 24 cores and 2 threads per core, totaling 2 NUMA nodes. Each socket is equipped with three memory channels, housing 16 GB DDR4 DIMMs and 128 GB Optane PMMs. In aggregate, the system comprises 192 GB of DRAM and 1.5 TB of Optane persistent memory. To mitigate the NUMA effect, one socket is designated for running the NVM Middleware threads, while the other handles the interactive and batch applications, as described in Section 3.4.3.

\subsection{Optane DC PMem Configuration}
As outlined earlier, this thesis concentrates on exploring the persistent capabilities of Optane DC PMem. Consequently, Optane DC PMem is employed in the App Direct Mode throughout our experiments. To facilitate the utilization of persistent memory, we expose it via an xfs filesystem configured in dax mode, thereby bypassing the page cache. Additionally, we enhance memory management and performance by configuring the persistent memory with huge pages (2MiB) \cite{Speeding28:online}. Lastly, we deploy a PMEMKV database with a capacity of 600GB, configured with its persistent concurrent engine.

\subsection{Workload Traces}

To execute the interactive and batch applications described in Section 3.4.3, we gathered logs from two serverless platforms.

For the interactive application, we utilized traces collected from Azure Functions. The dataset, available in \cite{GitHubAz35:online}, offers a comprehensive record of Azure Function blob accesses spanning November to December 2020. Our focus was on requests occurring between November 1 and November 5, particularly those with small data access sizes (less than 1 KB), which are indicative of interactive applications.

For the batch application, we gathered traces from Wukong, a serverless parallel computing framework \cite{carver2020wukong}. Traces were obtained by executing a Single Value Decomposition for a 128x128 matrix on Wukon and capturing the I/O requests generated by the framework. These collected traces represent the behavior of a throughput-oriented serverless data-analytics application.

To further amplify the concurrency of requests directed to the NVM Middleware, we accelerated the pace of the traces by a factor of 5 compared to their original timing.

\section{Fixed Concurrency Control}

\section{Dynamic Concurrency Control}

\subsection*{Pre-Training Linear Regression Models}

\subsection*{RL Training}

\subsection*{RL in Action}
